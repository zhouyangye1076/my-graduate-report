\section{问题提出的背景}

\subsection{背景介绍}

现代处理器为了追求极致的硬件性能,引入了大量复杂的投机执行技术\cite{gonzalez2010processor}。这些技术会对未执行完毕的指令结果进行预测,便于后续指令可以提前执行,从而提高处理器执行效率和硬件利用率。一旦发生预测错误,部分本不该被执行的指令就会被临时执行,这部分指令被称为瞬态执行指令。当预测错误被发现时,处理器将回滚瞬态执行指令的执行结果,从而确保处理器执行的正确性。但是瞬态执行指令造成的微架构状态的改变仍然会被保留,因此可以采取特殊攻击技术跟踪这些状态变化,并尝试从中恢复出处理器内部的安全敏感数据,从而违反了 CPU 的基本安全隔离保证。自Spectre\cite{kocher2020spectre}和 Meltdown\cite{horn2018meltdown} 等瞬态执行漏洞被披露以来,各类瞬态执行漏洞如 ForeShadow\cite{van2018foreshadow}、CacheOut\cite{van2021cacheout}、Ret2Spec\cite{maisuradze2018ret2spec}、RIDL\cite{mathure2023hardware}、MDS\cite{minkin2019fallout}、FPVI\cite{ragab2021rage}不断涌现——瞬态执行漏洞已经成为现代cpu中的关键漏洞。且因为瞬态执行漏洞利用最底层的硬件漏洞发动攻击,所以任何运行在 CPU 上的实体,无论其特权级和工作模式,都有可能成为攻击者和受害者。这对处理器软件安全造成了重大威胁。\par

虽然这些漏洞严重损害了 CPU 上软件的安全性,但受制于硬件漏洞的不可修复性,各大厂商无法为已经出厂的芯片提供漏洞修复的办法,因此在 RTL 开发阶段发现瞬态执行漏洞,并及时加以修复,就显得至关重要。随着软件模糊测试技术的不断成熟,该技术也被逐步应用于处理器漏洞挖掘,并且已经发现了很多具有影响力的硬件功能性漏洞。但是如何构造针对瞬态执行漏洞来自动化生成有效且多样化的测试程序,仍然是瞬态漏洞模糊测试领域亟待解决的重要课题。\par

\subsection{本研究的意义和目的}

我们的工作旨在挖掘开源处理器的瞬态执行漏洞。瞬态执行漏洞具体来说就是由特定指令序列组合起来引发的 bug,这些指令会导致处理器触发瞬态执行,并通过侧信道将处理器机密信息泄露出去,从而导致处理器行为违反基本安全隔离保证。通过处理器模糊测试的方法可以帮助处理器尽早的在 RTL 开发阶段检测和修复并修复瞬态执行漏洞,具有重大意义。但是因为瞬态执行漏洞具有其复杂性和特殊性,传统的处理器模糊测试的测试程序生成方法因为没有充分结合瞬态执行漏洞的特点,并不能高效的生成高质量的漏洞触发程序,导致其在实际应用中能力受限。\par

因此,我们的工作将尝试结合瞬态执行漏洞的语义特点,构建一个高效的自动化测试程序生成框架,以期提高处理器瞬态执行漏洞挖掘的效率。我们还会尝试将该测试程序生成框架与其他处理器模糊测试工具相结合,评估该框架的执行效果,并通过挖掘一些开源 RISCV 处理器隐含的瞬态执行漏洞来证明框架的有效性和现实价值。\par





