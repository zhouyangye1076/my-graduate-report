\section{研究计划进度安排及预期目标}

\subsection{进度安排}

3 月 25 日:提交开题报告\par
4 月 2 日:确定环境配置参数,完成环境初始化模块\par
4 月 7 日:完成测试程序生成框架,并提供各功能块的默认模板\par
4 月 10 日:基于 BOOM 处理器完成后端处理器的对接\par
4 月 13 日:修改 spike 模拟器,集成测试程序过滤功能\par
4 月 20 日:为各功能块定制生成策略,结合后端测试调整生成策略,开始毕业论文写作\par
4 月 30 日:尝试通过生成策略进行 BOOM 瞬态漏洞挖掘,同时调整生成策略\par
5 月 3 日:选择若干开源 RISC-V 处理器进行后端处理器对接\par
5 月 12 日:与 SpecDoctor 进行瞬态漏洞挖掘效率、瞬态漏洞挖掘种类等对比实验,利用框架尝试发现新的处理器瞬态漏洞,并进一步完善框架\par
5 月 21 日:提交毕业论文\par

\subsection{预期目标}

实现瞬态执行漏洞测试生成框架,完成 64 位下 M/S/U 三种特权态下的环境搭建,支持虚拟地址测试。在瞬态漏洞挖掘效率方面超过 SpecDoctor。能发现一些开源处理器已有的瞬态执行漏洞,并期望能发现一些新的瞬态执行漏洞。
