\cleardoublepage

\section{背景和相关工作}

本章节我们对瞬态执行漏洞测试程序生成的相关背景和工作进行介绍。背景部分我们将会对 RISCV 指令集架构、
瞬态执行漏洞、处理器模糊测试尤其是差分模糊测试等技术进行介绍。在相关工作部分,我们则会对国内外瞬态执行漏洞挖掘
的测试程序生成、测试执行和漏洞检测等技术进行介绍。\par

\subsection{RISC-V指令集架构}

RISC-V 指令集架构(ISA)是一种开源的精简指令集架构\cite{riscv-isa-manual-all}。
它由一个基本的整数指令集和一组可选的指令集扩展组成。
标准扩展包含整数乘除法扩展(M)、内存原子操作扩展(A)、单精度浮点扩展(F)、双精度浮点扩展(D)和压缩指令扩展(C)等。
此外,控制和状态寄存器指令扩展(ZICSR)提供特权态管理功能, 而屏障指令扩展(ZIFENCEI)提供指令内存同步功能。\par

RISC-V 指令分为 32 位常规指令(包含六种指令格式)和 16 位压缩指令(包含九种指令格式)两种。 
图\ref{review:base-inst}和图\ref{review:compress-inst}显示了所有 15 种指令类型的格式,
每种指令由多个字段组成。32 位指令的指令格式由 opcode 字段决定,而 16 位指令的格式由 op 和 funct 字段同时决定。\par

\begin{figure}[!h]
    \centering
    \includegraphics[width=\linewidth]{figure/proposal/riscv-base-instruct-format.png}
    \caption{RISCV基本指令格式}
    \label{review:base-inst}
\end{figure}
\begin{figure}[!h]
    \centering
    \includegraphics[width=\linewidth]{figure/proposal/riscv-compress-instruct-format.png}
    \caption{RISCV压缩指令格式}
    \label{review:compress-inst}
\end{figure}

指令字段可分为两类:
\begin{enumerate}
    \item 操作码相关字段,如 funct 和 opcode 字段。
opcode 字段和 op 字段的低 2 位用于确定指令的长度,而 funct 字段和 opcode 字段的其它位用于确定指令格式和功能。
通常,具有相似功能的指令有相同的 opcode 字段,并通过 funct 字段来区分具体功能。\par
    \item 操作数相关的字段,包括 imm、rs 和 rd 字段。
其中 rs 字段用于选择源寄存器,imm 字段用于表示立即数,rd 字段用于选择目的寄存器,即写回指令结果的寄存器。\par
\end{enumerate}

\subsection{瞬态执行攻击}
瞬态执行漏洞是现代处理器的关键问题。
现代处理器为了追求高性能,广泛采用乱序执行和投机执行等技术提高硬件利用率。
由于异常延迟处理和推测错误等原因,部分指令会被错误执行,虽然其执行结果会在后续被撤销,
并不会影响处理器架构层执行结果的正确性,但仍可能引起处理器微架构的状态变化。
攻击者可以通过侧信道跟踪微架构的状态变化,进而恢复出处理器内部的机密信息,这种攻击方式称为瞬态执行攻击。\par

瞬态执行攻击过程可以分为如下三个阶段。\par

\textbf{触发瞬态执行窗口:}
通过操纵控制流和数据流,可以人为创造异常延迟检查或者推测错误等场景,触发瞬态执行窗口,从而临时性地绕过软硬件的安全权限检查。
例如 Meltdown 类型利用了延迟的权限检查,如延迟对页表条目中标志位的检查\cite{horn2018meltdown},
或对浮点寄存器权限的检查\cite{stecklina1806lazyfp},暂时绕过硬件原语的安全权限检查,瞬态执行后续的代码;
而 Spectre 类型\cite{kocher2020spectre} 则通过毒害返回地址堆栈\cite{maisuradze2018ret2spec}、分支预测器等,
操纵控制流进入错误预测的跳转地址,从而暂时绕过软件的安全权限检查,瞬态执行错误预测地址之后的代码。\par

\textbf{访问秘密数据:}
在第一阶段触发瞬态执行窗口之后,处理器可以临时性地绕过软硬件的安全权限检查,执行没有安全权限的代码。
通过执行这些代码可以直接或者间接访问内存中的秘密数据,
将秘密数据暂时从存储部件泄漏到寄存器中\cite{van2019ridl}\cite{van2021cacheout}。
这里可以直接访问秘密数据的地址,将内存、缓存、line fill buffer 中的秘密数据泄漏到寄存器中;
也可以访问低位对齐等相关的地址,将存储层次中的秘密数据泄漏到寄存器。

\textbf{使用侧信道泄露秘密数据:}
将秘密数据带入寄存器之后,通过微体系结构的微架构状态对秘密数据进行编码。
之后通过特殊的微架构跟踪技术,通过时间侧信道等侧信道,将编码在微架构状态中的秘密数据解码出来,从而实现秘密数据的侧信道泄露。
已知的瞬态执行漏洞往往通过缓存\cite{yarom2014flush+}、分支预测器\cite{evtyushkin2018branchscope}、
执行端口\cite{bhattacharyya2019smotherspectre}等微架构部件进行秘密数据的编码,
然后通过时间侧信道将编码在其中的信息泄露出来。
随着 CPU 复杂度的不断提高和瞬态攻击研究的不断推进,更多微架构状态和侧信道被应用于瞬态执行攻击。\par

\subsection{处理器模糊测试}

模糊测试是一种自动化测试技术,它会根据规则自动或者半自动地生成随机数据,
然后输入到动态运行的被测程序入口,通过监控被测程序是否有异常情况发生(如系统崩溃,断言失败等)来发现存在的软件缺陷。
同时一些模糊测试生成器会根据被测程序插桩的信息反馈,有指向性地对输入数据进行突变,
进一步提高异常触发的效率和测试程序的覆盖率。
随着处理器复杂程度的提高,传统的处理器验证方式效果开始不断受限,
研究人员转而将模糊测试技术引入到处理器正确性验证领域 
\cite{bruns2022efficient}\cite{canakci2021directfuzz}\cite{hur2021difuzzrtl}。\par

处理器模糊测试和软件模糊测试一样也由三个阶段组成:\par

1.输入生成阶段:
模糊器使用种子生成指令流,并根据前一轮的覆盖范围对指令流进行突变,
如 DifuzzRTL\cite{hur2021difuzzrtl}使用静态分析技术生成指令,
而 TheHuzz\cite{kande2022thehuzz} 则使用最优权重优化算法进行突变。\par

2.硬件仿真阶段:
模糊测试工具使用硬件插桩技术来收集当前输入的覆盖率,现有的模糊测试工具设计了多种覆盖度量,
诸如 mux 覆盖\cite{laeufer2018rfuzz}、控制寄存器覆盖\cite{hur2021difuzzrtl}和硬件行为覆盖\cite{kande2022thehuzz} 等。\par

3.状态验证阶段:
模糊测试工具提取 DUT 的架构状态,然后将其与参考模型(例如,ISA 模拟器)进行比较,其中不匹配的行为将被标记为 bug。\par

\par

\textbf{差分模糊测试:}
差分测试需要一种检测机制来检测错误或者说检测异常行为。
虽然一些简单的错误如内存损坏、段错误等很容易被检测器捕获,但那些难以定义明确异常行为的语义错误则很难被检测到。
因此,差分测试被用于识别语义相关的 bug(如文件系统\cite{min2015cross}、JVM\cite{chen2019deep} 等),
它通过比较具有相同功能和执行目标的多个程序的执行结果进行错误检测,
如果这些程序显示的输入-输出关系不同,则可以认定存在一个bug。\par

瞬态执行漏洞因其语义的复杂性,难以用简单的逻辑约束进行表征和检测,
而差分测试正好适用于语义相关 bug 的识别,故而也被广泛运用于瞬态执行漏洞的检测。
具体方法如图\ref{paper:differential-test}所示。
将两个仅秘密数据部分不同的可执行程序交给两个同样的处理器进行执行,如果两者执行的时间存在差异,
则认为可以通过时间侧信道对秘密数据进行泄露。\par

\begin{figure}[!h]
    \centering
    \includegraphics[width=\linewidth]{figure/paper/differential-test.png}
    \caption{差分测试流程}
    \label{paper:differential-test}
\end{figure}

\subsection{国内外相关工作}

瞬态指令漏洞的模糊测试检测作为处理器模糊测试的变体,也可以分为输入生成阶段、测试执行阶段和漏洞检测阶段三部分。
本章节将从这三个方面对研究方向和研究进展进行介绍。

\subsubsection{输入生成}
输入生成阶段负责生成瞬态执行漏洞测试程序。
早期的硬件模糊测试工具如 DifuzzRTL\cite{hur2021difuzzrtl}、
TheHuzz\cite{kande2022thehuzz}等基本采用纯随机的方式进行测试指令生成,
处理器漏洞挖掘效率相对较低。为了进一步提高测试程序的质量,Razzle\cite{razzle}、
Cascade\cite{soltcascade}等工具会对程序的控制流进行约束,以此提高程序执行的覆盖率;
SpecDoctor\cite{hur2022specdoctor}等工作对指令的数据流进行了约束,
通过提高相邻指令的数据依赖,提高其在处理器微架构执行时的难度。\par

为了充分利用瞬态执行漏洞的语义特点,一些测试程序生成工具采用基于模板突变的方法,
试图高效地产生瞬态执行漏洞的变体。如Transynther\cite{moghimi2020medusa}是专注于 MDS 变体生成的工具,
它使用已知的MDS构建块和微代码进行组合,来构造新的 MDS 泄漏,
而SpeechMiner \cite{xiao2019speechminer}则专注于 Meltdown 变体的生成。
但这种方法可能会导致漏洞同质化,缺乏挖掘全新漏洞的能力。\par

\subsubsection{测试执行}

测试执行阶段负责让后端处理器执行前端生成的测试程序,尝试在执行中触发瞬态执行漏洞。
早期的工作会直接使用真实的处理器硬件执行测试程序,但是随着RTL漏洞检测重要性的提高,
基于RTL仿真的瞬态漏洞测试执行工作,如IntroSpectre\cite{ghaniyoun2021introspectre}、
SpecDoctor\cite{hur2022specdoctor}等开始兴起。
verilator\cite{snyder2013verilator}等开源工具为RTL仿真提供了条件,
且在仿真的场景下还可以进行硬件插桩,便于后续阶段的漏洞检测。\par

但为了提高测试执行的效率,一些工作如Revixor\cite{oleksenko2022revizor}、
Scam-V\cite{nemati2020validation}、
Speculation at Fault\cite{hofmann2023speculation}等会对处理器硬件建立形式化模型,
然后基于形式化模型进行测试执行,其中Revixor\cite{oleksenko2022revizor}支持 x86 指令,
Scam-V\cite{hofmann2023speculation}支持 arm 指令,
Speculation at Fault\cite{hofmann2023speculation}则是对Revixor\cite{oleksenko2022revizor}
在异常相关类型漏洞上的一个补充。此外基于模型的测试方式也可以对模型直接进行瞬态执行漏洞的静态分析,
如UPEC\cite{fadiheh2020formal}根据开源RTL手动建立有界模型,然后利用静态分析技术寻找瞬态漏洞。

\subsubsection{漏洞检测}

漏洞检测阶段负责对测试执行结果进行分析,进而判断是否触发了瞬态执行漏洞。
差分测试是广泛使用的瞬态执行漏洞检测方法,Revixor\cite{oleksenko2022revizor}、
Speculation at Fault\cite{hofmann2023speculation}等工作都使用差分测试进行瞬态执行漏洞检测,
SpecDoctor\cite{hur2022specdoctor}更是在两个不同的阶段分别运用了差分测试。\par

此外还可以用基于约束的方法进行瞬态执行漏洞的检测,如Transynther\cite{moghimi2020medusa}、
IntroSpectre\cite{ghaniyoun2021introspectre}通过检查缓冲区执行前后的变化进行漏洞检查,
SpeechMiner\cite{xiao2019speechminer}通过测量竞争条件判断漏洞的可用性。\par

差分测试和约束判断的漏洞检测方法有时还需要硬件插桩的辅助。
如果拥有 RTL 源码,就可以通过 RTL 硬件插桩的方法为检测程序提供处理器内部的状态信息和事件信息,
甚至可以在处理器内存内嵌约束判断的电路逻辑,在测试执行的同时直接进行漏洞约束的判断。

\cleardoublepage
\section{研究挑战}

在生成瞬态执行漏洞测试程序时,生成框架会遇到三个重要挑战。
首先,在指令生成层面,
生成框架需要能生成满足特定条件的指令操作数和指令间依赖,
以满足特定的数据流和控制流约束;
其次,为了高效地生成能触发瞬态窗口的代码,
生成框架需要有意识地为触发瞬态窗口的指令设计和排布训练代码;
最后,为了生成有效的测试程序,生成框架需要让程序的各个语义部分可以相互适配,
协同工作,因此生成框架需要根据语义对测试程序代码块进行合理划分和针对性生成,
然后将各阶段代码块有效组合起来,最终删除冗余代码,得到精简的测试程序。\par

\subsection{带约束的指令生成}
\textbf{约束指令操作数}。
为了触发瞬态执行漏洞,生成框架生成的指令需要能执行特定的操作。
例如为了可以泄露秘密数据或者触发内存访问异常,
访存指令计算得到的操作数需要落在特殊的内存地址范围内;
为了让控制流指令可以跳转到预设的地址,跳转指令计算得到的跳转地址需要等于预设的地址、
分支指令操作数的值要满足对应的比较规则等。
为了确保指令在执行时可以得到满足要求的操作数,从而实现特殊的语义行为(如跳到特殊地址、访问特殊内存等),
在生成指令时需要对指令立即数字段、数据段值、指令组合等进行约束处理。\par

\textbf{约束指令字段}。
为了触发一些预设的特殊场景,生成框架也需要对生成指令的字段进行约束。
例如为了增加指令间的数据流依赖、触发更复杂的流水线执行情况,需要对指令间的 rs、rd 等字段进行约束;
为了可以跳转到特定的地址、访问特殊的内存,需要对指令和数据的 label 等进行约束。\par

\subsection{瞬态窗口训练的生成}
\textbf{训练代码的生成}。
Spectre 等瞬态执行漏洞指出,触发瞬态窗口除了需要触发瞬态窗口的 trigger 指令(如 Spectre 的控制流指令、
Meltdown 的内存访问指令等),还需要有指令序列对 trigger 相关的微架构部件的状态进行训练。
但是以往的工作如DifuzzRTL\cite{hur2021difuzzrtl}、TheHuzz\cite{kande2022thehuzz}、
SpecDoctor\cite{hur2022specdoctor}都是用随机代码生成的方式进行瞬态窗口触发代码的生成,
并没有有针对性地生成训练代码,这导致了瞬态窗口的生成效率低下。
故而生成器需要能根据 trigger 指令,有针对性地生成用于训练的指令序列,进而提高瞬态窗口触发的可能性。\par

\textbf{地址敏感的代码排布}。
指令序列对 trigger 的训练效果是地址敏感的。
在多数情况下,训练指令只有在特殊的地址上才可以对 trigger 指令对应的微架构状态进行有效训练 。
例如对于 boom 处理器\cite{celio2017boomv2}的 branch 指令,
低地址对齐的控制流指令可以对该 trigger 对应的分支预测器表项进行有效训练,
从而将 branch 指令涉及的微架构状态调整到位。
为此,生成框架在生成训练代码的时候需要寻找合适的指令排布位置,
从而提高训练的成功率。 \par

\subsection{完整程序的生成}

\textbf{程序的模块分割}。
已知完整的瞬态执行漏洞程序由触发瞬态窗口、访问秘密数据、侧信道泄露秘密数据三部分代码组成,
为了提高生成框架对瞬态执行程序语义的利用,生成框架可以将三阶段代码进行进一步细分,
然后根据各个代码块细分后的功能提供更有针对性的生成策略。\par

\textbf{程序的组合}。
为确保分割后的代码能够按场景依次执行、协同工作,从而组成完整有效的瞬态漏洞程序,
我们需要通过总结分析各个代码块之间的协作约束关系,包括操作数关系、控制流关系等,
设计出通用的代码组装模版,在确保瞬态漏洞程序有效运行的同时,尽可能多地覆盖各类瞬态漏洞场景。\par

\textbf{程序的精简}。
生成器得到的瞬态执行漏洞程序不可避免地包含了许多对瞬态执行攻击没有帮助的指令,
这些指令会增加程序模拟执行和漏洞检测的时间开销,并且也会对后续程序漏洞定位带来负担。
因此生成框架需要使用合适的算法精简掉冗余的指令,从而得到精简的瞬态漏洞程序。\par

\cleardoublepage
\section{测试程序生成框架的设计}

本生成框架是一种针对 RISCV 处理器 RTL 的、用于瞬态漏洞查找的测试程序生成框架。
对于给定的 RISCV 处理器 RTL 和代码生成配置,该生成框架可以根据瞬态执行漏洞各阶段的程序特点,
高效地生成瞬态执行漏洞测试程序,寻找发动瞬态执行攻击的具体 PoC,进而准确地复现漏洞。 \par

\subsection{生成框架概述}

为了高效地生成有效的瞬态执行攻击测试程序,我们的生成框架分两阶段进行测试程序生成。
在阶段1中,生成框架生成触发瞬态窗口的指令序列,并在处理器 RTL 上检验该部分指令的有效性;
阶段2在阶段1成功的基础上,生成访问秘密数据和利用侧信道传递秘密数据的指令序列,
如果该指令序列在处理器上存在差分测试差异,即可得到能触发瞬态执行漏洞的完整 PoC。\par

\subsection{阶段1:触发瞬态窗口}

不同于 SpecDoctor 先生成能触发瞬态窗口的随机指令序列,再确定 trigger 指令和瞬态窗口位置;
我们的生成框架预先确定待触发的瞬态窗口和触发该瞬态窗口的 trigger 指令,再为 trigger 指令生成协同工作
的训练代码和其他代码。之后该测试程序会在 vcs 或者 verilator\cite{snyder2013verilator} 生成的处理器 RTL 上模拟执行,
检查预设的瞬态窗口是否触发。如果触发成功,生成框架会删去冗余的训练指令,然后进入阶段2生成后续指令。\par

\subsubsection{瞬态窗口触发代码的生成与排布}

为了实现触发瞬态窗口的目的,生成框架会生成如下配合工作的代码块:\par

\textbf{全局初始化块(init块):}
init 块负责在程序执行伊始为所有特权寄存器赋初值,然后执行后续的瞬态窗口触发代码。\par
\textbf{寄存器初始化块(load\_init块):}
load\_init 块负责为后续所有被执行块所使用的通用寄存器和浮点寄存器提供初值。\par
\textbf{延迟块(delay块):}
delay 块在 load\_init 块初始化之后执行,它由一系列强数据流依赖的多周期指令组成,用于延迟得到操作数的计算结果。\par
\textbf{触发块(trigger块):}
trigger 块负责触发瞬态窗口,紧接着 delay 块开始执行。为了延迟得到 trigger 指令的执行结果,
迫使处理器执行预测执行,trigger 指令的操作数依赖于 delay 块的计算结果。
此外,为了可以让 trigger 指令可以访问特殊的地址触发异常、或者跳转到非瞬态窗口的地址,生成框架会对 trigger 指令的操作数值进行约束,
确保地址访问或者控制流跳转符合预期。\par
\textbf{瞬态窗口块(trans块):}
trans 块包含一个确保可以侧信道泄露秘密数据的默认指令序列,用于填充预设的待触发的瞬态窗口区域,
如果该部分的代码被瞬态执行,则预示着瞬态窗口被触发成功。\par

此外还有 return 块负责 trigger 块正确退出;trap 块负责作为异常处理程序;exit 块负责终止程序;nop 块负责区域填充等。
其中 init 块、return 块、exit 块、trap 块是因为测试框架需要所涉及的,因此称为功能块(function 块);
load\_init 块、nop 块、delay 块、trigger 块用于瞬态窗口的触发,因此称为触发块(trigger 块);
trans 块暂时填充阶段2用于秘密数据泄露的代码区域,因此属于泄露块(leak 块)。
根据瞬态漏洞触发的场景不同,生成框架用两种方式将上述代码块组织起来,如图\ref{paper:trigger-dist}所示。\par

首先执行 init 块,然后顺序排布和执行 load\_init 块、nop块、delay 块、trigger 块。
根据 trigger 指令的指令类别,决定后续排布的 return 块和 trans 块的顺序。trap 块位于其他位置,在异常发生时进入。
最后 return 块、trans 块、trap 块都会跳转进入 exit 块,结束硬件模拟。
场景一的排布可以满足控制流跳转、异常触发和 spectre-V4 等 trigger 类型的执行需求;
场景二的排布可以满足控制流不跳转、非跳转指令等 trigger 类型的执行需求。
这里的 load\_init 块、delay 块、trigger 块、nop 块、
trans 块、return 块组成了一个完整的指令组合,我们称之为 victim group。\par

\begin{figure}[!h]
    \centering
    \includegraphics[width=0.8\linewidth]{figure/paper/arrange-stage1.png}
    \caption{瞬态窗口触发代码块布局}
    \label{paper:trigger-dist}
\end{figure}

\subsubsection{训练代码的生成和排布}

为了 trigger 块的 trigger 指令可以更有效地触发瞬态窗口,生成框架需要生成额外的 train 指令对 trigger 指令进行训练,
因此生成框架在之前的基础上引入了用于训练的 train 块。train 块可以由单条或者多条任意类型指令随机组合得到,
对于 train 块中的控制流指令和内存访问指令,生成框架对跳转地址和访问地址做额外约束,确保程序能正确执行。\par

考虑到 trigger 指令相关部件的微架构状态训练是地址敏感的,
因此 train 块布局的地址最好和 trigger 的地址相关,如虚拟地址、甚至物理地址相同或者地址低位对齐等。
为了达到最高效的训练效果,也为了对物理地址空间和虚拟地址空间做最小的改动,
硬件额外扩展了一种物理内存切换机制,可以对物理地址对应的内存单元直接进行切换,
从而确保 train 块和 trigger 块可以先后在同一物理地址执行,进而提高 train 块训练成功的概率。
此外生成框架还会生成用于寄存器初始化的 load\_init\_train 块和作为 train 块控制流转移目标的
return 块、nop\_return 块等代码块,配合 train 块工作。\par

生成框架会用上述代码生成与 victim group 地址布局对应的训练代码,我们称之为 train group。
有时候一个 train group 并不能有效训练 victim group,因此生成框架会依概率为 victim group
生成一系列 train group,依次执行,来提高成功训练 trigger 指令的可能性。
如图\ref{paper:memory-switch}所示,当执行测试程序时,
首先执行第一个 train group 的指令序列,然后返回到物理内存切换的指令部分,
执行物理内存切换。物理内存中的指令切换为下一个 train group,继续训练,直到最后切换为
被训练的 victim group,开始尝试触发瞬态窗口。\par

\begin{figure}[!h]
    \centering
    \includegraphics[width=\linewidth]{figure/paper/group-switch.png}
    \caption{代码组内存切换}
    \label{paper:memory-switch}
\end{figure}

\subsubsection{测试程序的检验和简化}

为了判断阶段1生成的代码能否有效执行,生成框架生成的测试程序
需要在 vcs 或者 verilator 编译生成的处理器 RTL 上模拟执行,
检测是否触发了瞬态窗口,也即是否执行了 victim group 的 trans 块的代码。
我们在 trans 块插入特殊的 label 指令,如果硬件插桩检测到该指令被瞬态执行,
则说明阶段1代码生成成功。\par

在为 victim group 生成 train group 序列的时候,测试框架为了确保 trigger 指令被有效训练,
可能会生成过饱和的 train group 序列。其中有部分 train group 可能是无效的,或者冗余的。
为了减轻后续阶段2执行 train group 的开销,也为了更好地定位训练 trigger 指令的有效 train 片段,
生成框架需要能删去多余的 train group 代码。\par

\begin{algorithm}[!h]
    
    
    \caption{Train Simplificaiton}
    \label{alg:TS}
    \renewcommand{\algorithmicrequire}{\textbf{Input:}}
    \renewcommand{\algorithmicensure}{\textbf{Output:}}
    
    \begin{algorithmic}[1]
        \REQUIRE train group list $train\_list$, victim group $victim$  %%input
        \ENSURE train group list after simplification $new\_train\_list$    %%output
        \STATE $new\_train\_list \Leftarrow train\_list$
        \FOR{each $i \in [1, len(train\_list)]$}
            \FOR{each $j \in [1, len(new\_train\_list)$]}
                \STATE $tmp\_train\_list \Leftarrow new\_train\_list with new\_train\_list(i)$
                \IF{$trigger\_test(tmp\_train\_list, victim) == True$}
                    \STATE $new\_train\_list \Leftarrow tmp\_train\_list$
                    \STATE \textbf{break}
                \ENDIF
            \ENDFOR
        \ENDFOR

        \RETURN $new\_train\_list$
    \end{algorithmic}
\end{algorithm}

生成框架使用算法\ref{alg:TS}进行 train group 的化简。该算法尝试删去 train group 序列中的一个 train group,
然后重新执行处理器模拟执行,如果仍然可以触发瞬态窗口,尝试删除下一个 train group。
如果尝试删除任意一个 train group 之后都无法触发瞬态窗口,则化简结束,得到局部最优的 train group 序列。\par

\subsection{阶段2:访问和泄露秘密数据}

在阶段1得到的触发瞬态窗口代码的基础上,生成框架继续生成访问秘密数据和利用侧信道传递秘密数据的代码,
并通过处理器差分测试检测生成的测试程序能否最终触发瞬态执行攻击。
如果测试存在执行时间的差异,该测试程序即为有效的 PoC。

\subsubsection{访问秘密数据代码的生成与排布}

为了实现访问秘密数据的目的,生成框架会生成如下的代码块:\par

\textbf{秘密数据访问块(access\_secret 块):}
access\_secret 块的代码会生成待泄露的 secret 数据相关的地址,然后用 load 指令访问该内存,
将 secret 数据读取到寄存器中。鉴于访问秘密数据的过程只能在瞬态窗口中进行,
因此 access\_secret 块的代码需要被布局在 trans 块的内部。\par

\textbf{秘密数据迁移块(migrate\_secret块):}
migrate\_secret 块负责改变 secret 数据在分层存储结构中的位置,如 line fill buffer、cache、memory。
生成框架产生秘密数据迁移的代码,将秘密数据迁移到合适的存储深度,探索 access\_secret 块
访问不同深度秘密数据的能力。\par

\begin{figure}[!h]
    \centering
    \includegraphics[width=\linewidth]{figure/paper/stage2-access-secret.png}
    \caption{访问秘密数据代码的排布}
    \label{paper:access-secret}
\end{figure}

如图\ref{paper:access-secret}所示,
这两部分代码在阶段2会被组装进入 victim group。对于 access\_secret 块,
因为访问秘密数据的过程只能在瞬态执行的过程中进行,
因此 access\_secret 块的代码需要被布局在 trans 块的内部;
而对于 migrate\_secret 块,这部分代码虽然可以在瞬态窗口外部被执行,
但是在执行了 migrate\_secret 块的指令之后应该尽可能地避免内存访问指令的执行,
以防破坏 secret 在内存结构中的位置,因此这部分代码被放置在 load\_init 块之后、delay 块之前。
故而阶段2将原先的 victim group 调整为如图\ref{paper:access-secret}所示的布局。\par

当 victim group 的代码被执行时,程序在执行完 load\_init 块的寄存器初始化之后,
就会执行 secret\_migrate 块的代码,将秘密数据迁移到处理器存储结构的合适位置,然后执行后续代码触发瞬态窗口。
瞬态窗口被触发之后,access\_secret 块被执行,试图将被 secret\_migrate 块迁移之后的秘密数据
从存储层次结构读取到寄存器中。\par

\subsubsection{侧信道泄露代码的生成与排布}

为了实现侧信道泄露秘密数据的目的,生成框架会生成如下的代码块:\par

\textbf{编码块(encode 块):}
encode 块利用 access\_secret 块访问得到的 secret 数据,尝试修改处理器的微架构状态,
以此实现将 secret 数据编码到处理器微架构状态的目的。为了让 encode 块有较强的 secret 数据传播和编码能力,
生成的指令会尽可能多地使用 secret 数据,并让生成的指令之间存在较强的数据竞争和结构竞争。\par

\textbf{解码块(decode 块):}
decode 块尝试利用 encode 产生的微架构状态差异产生执行时间上的差异,
以此实现将 secret 在处理器微架构的编码利用时间侧信道解码的目的。
因为重复执行相同的指令会涉及到相同微架构部件的使用,故而一条指令被重复执行时,
其执行结果易受初次执行时产生的微架构状态的影响,
因此生成框架简单使用和 encode 块相一致的代码作为 decode 块。\par

\textbf{替换块(replace 块):}
replace 块用于替换 encode 块和 delay 块之间的部分,它直接用立即数指令生成 secret 数据的值,并将值存储到和 access\_secret 一致的目的寄存器中。
执行 replace 块可以得到和 access\_secret 块一样的执行结果,但是并不会修改 cache、内存等存储单元。

在进行代码排布时,encode 块需要在瞬态窗口中执行,因此被放置在 trans 块内部,且在 access\_secret 块执行完毕后执行;
decode 块为了可以和 encode 块使用尽可能一样的处理器部件,因此它除了指令和 encode 块高度相似外,
还需要地址和 encode 块保持一致,因此生成框架新定义一个 decode group,
将 decode 块排布在 decode group 中与 encode 块相同的物理地址处。\par

\begin{figure}[!h]
    \centering
    \includegraphics[width=\linewidth]{figure/paper/group-switch-all.png}
    \caption{完整程序内存切换}
    \label{paper:group-switch-all}
\end{figure}

decode group 在 victim group 之后执行,用于对 encode 块编码在处理器中的 secret 数据进行解码。
如图\ref{paper:group-switch-all}所示,完整程序在执行时首先执行 train group 序列进行 trigger 指令的训练,
然后切换到 victim group 触发瞬态窗口、访问秘密数据和将秘密数据编码到微架构状态,最后切换到 decode group,
尝试从微架构状态中解码 secret 的值。为了确保 decode group 执行 decode 块时可以获得和 encode 块执行时一样的操作数,从而使用一样的处理器部件,
decode group 复制 victim group 的 load\_init 块、delay 块;secret\_migration 块用 nop 块替代;
trigger 块、access\_secret 块和两者之间可能存在的 return 块用 replace 块代替,避免修改cache、内存的状态;decode 块替代 encode 块。
因此 decode group 最后的代码块排布如图\ref{paper:group-switch-all}中的 decode group 所示。\par

\subsubsection{秘密数据泄露检验}

我们采用差分测试的方法对攻击是否成功进行检验。
首先将上述的所有 group 和其他框架代码导出为完整的测试程序,在处理器 RTL 上模拟执行。
为了测量测试程序的执行时间,生成框架会在程序结尾插入特殊 label 指令,
然后对 RTL 进行内部插桩,当该 label 指令被执行时, dump 此时的处理器周期数。
之后将测试程序的 secret 部分替换掉,再次进行模拟执行和 dump 执行结束时的处理器周期数。
如果两者的执行时间存在差异就说明 secret 信息通过时间侧信道泄露出来了,瞬态执行攻击成功,得到完整 PoC。

\cleardoublepage
\section{测试程序生成框架的实现}

本工作由瞬态漏洞测试程序生成框架和硬件处理器插桩修改两部分组成。
其中测试程序生成框架包括 5.2 K 行 Python 和 300 行 RISCV 汇编,负责瞬态漏洞测试程序的生成、测试调用和化简;
硬件测试平台在开源 RISCV 处理器测试框架 starship\cite{starship} 的基础上进行修改,
包括 650 行 C 代码、200 行 Verilog、50 行 chisel,用于提供硬件插桩和测试功能扩展。
我们的测试程序生成框架可以支持 U、S、M 三特权级的测试程序生成,可以支持虚拟地址、物理地址的测试程序生成,
在指令生成方面支持 RISCV 的 rv64imafdc 扩展的全集和 zicsr 扩展的子集,
可以高效生成 spectre、meltdown 等典型的瞬态漏洞测试程序。\par

\subsection{硬件插桩和功能扩展}

为了让被测试的处理器满足瞬态漏洞测试程序执行和测试的需要,我们需要对被测试的处理器硬件进行插桩和功能扩展,包括:
a)提供事件监控的软件 label 指令插桩和硬件 rob 插桩;b)提供处理器停机、结束测试程序的软件接口;
c)提供支持物理内存切换的内存模块和触发物理内存切换的软件接口。\par

我们在开源的 RISCV 计算机系统评估框架 starship\cite{starship} 的基础上进行处理器硬件的修改和测试。
starship 通过集成一系列开源的 RISCV 软硬件工具,实现自动化的全栈计算机系统生成,
可以支持 rocket-chip\cite{rocket-chip}、boom\cite{celio2017boomv2}、CVA6\cite{cva6} 等开源 RISCV 处理器的仿真、评估、FPGA执行等。
本工作目前仅在 starship 提供的 boom flow 上进行测试和评估,
故而下文的硬件插桩和处理器测试仅针对 boom 处理器而言。\par

\subsubsection{事件监控和 rob 插桩}
\textbf{label 指令:}
为了了解测试程序在处理器内部的执行情况,检查特殊事件(如瞬态窗口触发、程序执行完毕等)是否发生,
我们预留了一组 slti 指令作为特殊的 label 指令,并将它们排布在软件的合适位置作为软件插桩。
该部分指令如果被处理器执行,则意味着特殊事件发生。
各 label 指令表示的事件、排布位置等如表\ref{table:label-inst}所示,
因此在生成各个代码块和将它们组织为 group 的时候,需要在指定位置额外插入这些 label 指令。\par

\begin{table}[h!]
    \begin{center} 
    \caption{label指令} 
    \label{table:label-inst}  
    \resizebox{1.0\linewidth}{!}{
        \begin{tabular}{|l|l|l|l|} 
            \hline
            \textbf{label名称} & \textbf{汇编指令} & \textbf{排布位置} & \textbf{表示事件}\\
            \hline
            INFO\_VCTM\_START  &  slti zero, zero, 0  & group代码块的第一条代码  & group 开始执行            \\
            INFO\_VCTM\_END    &  slti zero, zero, 1  & group代码块的出口位置    & group 执行完毕            \\
            INFO\_DELAY\_START &  slti zero, zero, 2  & delay块的第一条代码      & 开始数据流延迟执行        \\
            INFO\_DELAY\_END   &  slti zero, zero, 3  & delay块的最后一条代码    & 瞬态窗口关闭              \\
            INFO\_TEXE\_START  &  slti zero, zero, 4  & encode块的第一条代码     & 瞬态窗口触发打开          \\
            INFO\_TEXE\_END    &  slti zero, zero, 5  & encode块的最后一条代码   &                          \\
            INFO\_LEAK\_START  &  slti zero, zero, 6  & decode块的第一条代码     & 开始进行秘密数据泄露       \\
            INFO\_LEAK\_END    &  slti zero, zero, 7  & decode块的最后一条代码   & 秘密数据泄露完毕           \\
            INFO\_INIT\_START  &  slti zero, zero, 8  & init块的第一条代码       & 开始寄存器初始化           \\
            INFO\_INIT\_END    &  slti zero, zero, 9  & init块的最后一条代码     & 寄存器初始化完毕           \\
            INFO\_BIM\_START   &  slti zero, zero, 10 & init\_bim块的第一条代码   & 开始等待 BIM 部件初始化   \\
            INFO\_BIM\_END     &  slti zero, zero, 11 & init\_bim块的最后一条代码 & BIM 部件初始化完毕        \\
            \hline
        \end{tabular}
    }
    \end{center}
\end{table}

\textbf{rob 插桩:}
我们需要对被测试处理器的 ROB 进行硬件插桩,通过监视测试程序中的 label 指令是否进入 rob 和
是否在 rob 中被提交或者回滚,判断瞬态窗口触发、瞬态窗口关闭、程序执行结束等事件是否发生,并记录这些事件发生的时间。
其中瞬态窗口触发事件的发生被用于阶段1的判断,程序执行结束的时间被用于阶段2差分测试的判断。
12 种 label 指令分别代表一类事件,根据 label 是进入 rob 还是从 rob 中提交可以分为 ENQ 和 DEQ 两种子事件,所以一共有 24 种事件。\par

首先创建一个记录事件信息的 taint.log 文件,然后实例化两个 event\_handler 函数,
根据指令编码分别对指令进入 rob 的 ENQ 事件和 rob 提交指令的 DEQ 事件进行记录,
包括事件名称(如 TEXE\_START\_ENQ、TEXE\_END\_DEQ)和当前处理器 cycle 数。

\subsubsection{软件接口和内存切换}

\textbf{软件接口:}
为了测试程序可以触发基于硬件的程序终止和内存切换,我们对 boom 处理器硬件进行修改并提供软件接口。
这里我们为 boom 处理器扩展了一个编号为 0x800 的用户态只写特权寄存器 probebuffer,
当 csrw 指令写 0x800 特权寄存器时即可触发对应的硬件功能。
接口的具体调用细节如表\ref{table:probebuffer}所示。\par

\begin{table}[h!]
    \begin{center} 
    \caption{probebuffer 软件接口} 
    \label{table:probebuffer}  
    \resizebox{0.6\linewidth}{!}{
        \begin{tabular}{|l|l|} 
            \hline
            \textbf{输入} & \textbf{功能}\\
            \hline
            0xAF1B608E883A0000 & probebuff切换到打印状态 \\
            0xAF1B608E883B0000 & probebuff执行 exit 函数结束仿真进程 \\
            0xAF1B608E883C0000 & 内存单元进行物理内存切换     \\
            其他               & 输出写入数据 \\
            \hline
        \end{tabular}
    }
    \end{center}
\end{table}

\textbf{内存切换单元:}
为了提高物理内存切换的效率,我们实现了可切换内存单元,用于各个 group 的快速切换。
该物理内存使用 DPI-C 接口调用 cpp 程序实现,函数接口如图\ref{code:swappable-memory}所示。

\begin{figure}[htbp]
    \centering
    \begin{minted}{cpp}
        void initial_mem(size_t mem_start_addr, size_t max_mem_size, std::vector<int>& schedule_list);
        void do_mem_swap();
        void write_byte(size_t addr, uint8_t data);
        uint8_t read_byte(size_t addr);
    \end{minted}
    \caption{可切换内存单元}
    \label{code:swappable-memory}
\end{figure}

\begin{enumerate}
    \item 内存初始化:生成框架会将各个 group 和其余框架部分打包为若干二进制文件和一个 libconfig 格式的 config 文件。
可切换内存单元调用 initial\_mem 函数,根据 config 文件的指示将二进制文件载入对应的内存区域。
其中框架部分的二进制会被物理地址直接映射,而 group 的二进制需要在执行内存切换后才会依次被物理地址映射。\par
    \item 内存读写操作:内存单元的读写操作调用 read\_byte、write\_byte 函数执行,
找到可切换内存单元中对应的字节,然后进行读写操作。\par
    \item 内存切换操作:内存单元的切换操作通过 do\_mem\_swap 函数执行。
根据 config 文件中的切换顺序,解除上一个 group 的二进制和物理地址的映射关系,
并将下一个 group 的二进制和物理地址相映射。\par
\end{enumerate}

\textbf{生成框架协同:}
参照图\ref{paper:group-switch-all},测试程序的执行流程是先执行初始部分代码,然后用内存切换机制依次执行 train group 序列、
victim group、decode group,然后执行 exit 块的代码退出。为了简化 group 之间的跳转流程,生成框架将 exit 块建模为
 exit group,作为最后一个被切换的 group。这样随着所有 group 切换完毕,仿真程序就会自动终止。\par

测试程序包括框架代码和各个 group 各自的代码,
但是因为各个 group 共用同一个物理地址,这导致单次生成的 elf 无法直接包含所有 group 的内容。因此生成框架每次只用一个 group 的
汇编代码和框架代码组合生成一个二进制文件,然后根据程序的符号表,将二进制文件切割为框架代码的二进制部分和 group 代码的
二进制部分。这些二进制的组织关系被记录在 config 文件中。\par

为了方便测试程序在执行的时候进行内存切换,生成框架将内存切换调用放置在 trap 块,然后将 group 的出口指令修改为 ebreak 指令。
确保 group 退出后进入 trap 块执行内存切换。trap 块末尾将返回地址修改为 group 的入口地址,退出异常处理程序,即可开始执行下一个 group。
内存切换流程如图\ref{paper:execute-flow}所示。\par

\begin{figure}[!h]
    \centering
    \includegraphics[width=\linewidth]{figure/paper/execute-flow.png}
    \caption{程序执行和内存切换流程}
    \label{paper:execute-flow}
\end{figure}

\subsection{框架组件生成}

生成框架不但要生成用于触发瞬态窗口和侧信道泄露秘密数据的 PoC 代码,
还需要额外生成一些框架代码,用于支持 PoC 代码的执行和生效。
本章节将对这些框架相关的代码部分、数据部分、其他机制进行罗列和详细介绍。\par

\subsubsection{框架代码块和框架数据块}
本章节将对框架相关的代码部分、数据部分进行罗列和详细介绍。

\textbf{全局初始化块(init 块):}
init 块用于对所有 GPR、FPR、CSR、PC 和特权级进行初始化。为了对寄存器,特别是特权寄存器进行 field 粒度的属性设置,
本工作实现了 snapshot 子模块。该模块会读入一个 reg\_init.hjson,这个文件记录了通用寄存器 GPR、浮点寄存器 FPR、
M态和S态的基本特权寄存器 field 粒度的属性值,然后拼接 CSR 的各个 field 得到 CSR 的初始值,并生成寄存器初始化的汇编代码。
该汇编代码将所有 CSR、FPR、GPR 的初始化值存储为一个 64 位数据的数组,然后依次读取数组的值来初始化 CSR、FPR、GPR。
生成框架的 InitManager 负责修改 reg\_init.hjson 各个字段,为每个特权寄存器提供满足测试程序要求的值,
并调用 snapshot 模块生成对应的初始化程序。例如 mtvec 和 stvec 需要设置为 trap 块的入口地址,satp 设置为页表基地址,
mstatus 的 pmie 设置为程序执行所需要的特权级,mepc 设置为后续程序执行的入口地址,然后在 init 块末尾执行 mret,
进入后续测试程序的入口,并自动设置程序执行时的特权级。\par

\textbf{异常块(trap块):}
针对 M 态、S 态的异常处理程序,生成框架分别生成 mtrap 块和 strap 块,用于 group 的退出和物理内存的切换。
当一个 group 执行完毕后就会进入 trap 块执行内存切换,
之后执行 fence 指令刷新 pipeline、icache 和 tlb 的指令缓存,完成处理器内部的指令同步。
之后将返回地址设置为 group 的入口地址(所有 group 入口地址保持统一)后返回下一个 group 入口进行执行。为了可以监控 group 开始执行和
执行结束两个事件的发生,生成框架在 trap 的入口插入一条 INFO\_VCTM\_END 表示 group 执行结束,
在 sret/mret 前插入一条 INFO\_VCTM\_START 表示 group 开始执行。\par

\textbf{退出块(exit 块):}
exit 块被用于测试程序的退出。
该模块将 0xAF1B608E883CB0000 写入 probebuffer 寄存器,结束程序。\par

\textbf{秘密数据块(secret 块):}
secret 块用于存放秘密数据,是瞬态漏洞执行程序泄漏的目标。secret 被初始化为一些魔数,并且生成框架确保测试程序只会在
 access\_secret 块和 secret\_migrate 块中访问 secret 数据,确保程序不在其他位置直接或者瞬态访问 secret 数据。
当执行差分测试时,变体测试程序会将 secret 部分替换为全 0,然后进行后续的差分测试。\par

\textbf{信道数据块(channel 块):}
channel 块是预留给 cache 微架构状态编码的内存区域。channel 由 256 个 64 字节组成,每 64 字节是一个 cacheline 的长度,
可以使用 access\_secret 得到的 secret 值作为访问 channel 数组的索引,从而将 secret 编码到 cache 中。\par

\textbf{随机数据块(random\_data 块):}
random\_data 块是两页随机数据。train 块、encode 块、trigger 块等代码块有时会执行随机内存访问,则生成框架可以让这些内存
读写指令访问 random\_data 块的数据,这样既不会破坏其他重要数据区域,又可以为数据流引入新的随机数,增加执行流的复杂程度和覆盖率。\par

\textbf{无用数据块(dummy\_data 块):}
dummy\_data 块内部的数据值没有意义,该块由四页内存空间组成,secret\_migrate 块通过访问 dummy\_data 块来调整 cache 的状态。\par

\textbf{访问异常数据块(access\_fault\_data 块):}
access\_fault\_data 块提供了一页被 pmpaddr、pmpcfg 设置为禁止访问的数据页。生成框架如果想要产生一个 load\_access\_fault 异常或者
store\_access\_fault 异常就可以让 load、store 指令访问这个页,然后触发异常。\par

\textbf{页错误数据块(page\_fault\_data 块):}
page\_fault\_data 块提供了一页被页表设置为禁止访问的数据页。生成框架如果想要产生一个 load\_page\_fault 异常或者
store\_page\_fault 异常就可以让 load、store 指令访问这个页,然后触发异常。\par

\textbf{框架数据块(data\_frame 块):}
data\_frame 块用于存放框架部分代码的数据,比如 exit 块要访问的数据。\par

\textbf{训练数据块(data\_train 块):}
data\_train 块用于存放所有 train group 需要使用的数据,例如每个 train group 的 load\_init\_train 块的寄存器初始化数据。
理论上让每个 group 的数据跟着指令一起做切换是最简洁,但是因为 dcache 的数据无法被刷新,这就导致如果数据物理内存发生切换,
并没有高效的方法让 dcache 中的数据进行同步,因此只能让 group 的数据分别使用不相交的地址空间。\par

\textbf{受害数据块(data\_victim 块):}
data\_victim 块用于存放 victim group 需要使用的数据,例如 victim group 的 load\_init 块的寄存器初始化数据。\par

\textbf{解码数据块(data\_decode 块):}
data\_decode 块用于存放 decode group 需要使用的数据,例如 decode group 的 load\_init 块的寄存器初始化数据。\par

\subsubsection{地址分布}

本章节我们介绍生成框架的地址分布和地址管理。生成框架下处理器内存的物理地址范围是 0x80000000-0x80040000,总计 64 KB 的内存范围。
因为生成框架生成的代码可能同时包含了 M 态、S 态、U 态执行的代码,对于不同特权态下运行的代码需要生成不同的逻辑地址。
对于 M 态的代码,其逻辑地址和物理地址保持一致,地址格式为 0x800xxxxx。对于 U 态的代码则在物理地址的基础上加一个常数偏移量作为逻辑地址,
地址格式为 0x000xxxxxxx。对于 S 态的代码因为页表项 U 位的设置,无法和 U 态的代码共用一份逻辑地址,因此加上另一个常数偏移量,地址格式为
0xfffxxxxx。\par

测试程序各个 section 的虚拟地址范围、物理地址范围和权限,各个 section 的代码块、
数据块组成如表\ref{table:section-dist}、\ref{table:group-dist}所示。

\begin{table}[h!]
    \begin{center} 
    \caption{section地址排布} 
    \label{table:section-dist}  
    \resizebox{1.0\linewidth}{!}{
        \begin{tabular}{|l|l|l|l|l|l|l|} 
            \hline
            \textbf{section名称} & \textbf{包含的block} & \textbf{M 模式基址} & \textbf{S 模式基址} & \textbf{U 模式基址} & \textbf{长度} & \textbf{权限} \\
            \hline
            .init       &   init 块     &   0x80000000  &   \textasciitilde &   \textasciitilde &   1K  &   物理模式     \\ 
            .pagetable  &   pgtlb块     &   0x80001000  &   \textasciitilde &   \textasciitilde &   3K  &   物理模式     \\
            .secret     &   secret块    &   0x80004000  &   0xfff04000      &   0x00004000      &   1K  &   URW             \\ 
            .channel    &   channel块   &   0x80005000  &   0xfff05000      &   0x00005000      &   4K  &   URW             \\ 
            .mtrap      &   mtrap块     &   0x80009000  &   \textasciitilde &   \textasciitilde &   1K  &   物理模式     \\
            .strap      &   strap块     &   0x8000a000  &   0xfff0a000      &   0x0000a000      &   1K  &   RWX             \\
            .text\_frame &   init\_bim块  &   0x8000b000  &   0xfff0b000      &   0x0000b000      &   1K  &   URX              \\
            .random\_data&   random\_data块&  0x8000c000  &   0xfff0c000      &   0x0000c000      &   2K  &   URW              \\
            .data\_frame &   data\_frame块&   0x8000e000  &   0xfff0e000      &   0x0000e000      &   1K  &   URW              \\
            .data\_frame &   data\_victim块&  0x8000f000  &   0xfff0f000      &   0x0000f000      &   1K  &   URW              \\
            .data\_frame &   data\_decode块&  0x80010000  &   0xfff10000      &   0x00010000      &   1K  &   URW              \\
            .data\_frame &   data\_train块&   0x80011000  &   0xfff11000      &   0x00011000      &   1K  &   URW              \\
            .text\_swap  &   见表\ref{table:group-dist}  &   0x80020000  &   0xfff20000      &   0x00020000      &   1K  &   URX              \\
            .dummy\_data &   dummy\_data块&   0x80038000  &   0xfff38000      &   0x00038000      &   4K  &   URW              \\
            .access\_fault\_data  &   access\_fault\_data块 &   0x8003c000  &   0xfff3c000  &   0x0003c000  &   1K  &   不可访问 \\
            .page\_fault\_data    &   page\_fault\_data块   &   0x8003d000  &   0xfff3d000  &   0x0003d000  &   1K  &   物理模式 \\
            .stack      &   stack块     &   0x8003e000  &   0xfff3e000  &   0x0003e000  &   2K  &  RW \\
            \hline
        \end{tabular}
    }
    \end{center}
\end{table}

\begin{table}[h!]
    \begin{center} 
    \caption{group的.swap\_text包含的代码块} 
    \label{table:group-dist}  
    \resizebox{0.9\linewidth}{!}{
        \begin{tabular}{|l|l|} 
            \hline
            \textbf{group名称} & \textbf{包含的block} \\
            \hline
            train group  & load\_init\_train 块,nop 块,train 块,nop\_return 块,return 块 \\
            victim group & load\_init 块,nop 块,secret\_migrate 块,delay 块,trigger 块,\\
                         & access\_secret 块,encode 块,return 块 \\
            decode group & load\_init 块,nop 块,delay 块,replace 块,decode 块,return 块 \\
            exit group   & exit 块 \\
            \hline
        \end{tabular}
    }
    \end{center}
\end{table}

生成框架的 PageTableManager 根据各个部分的地址分配和权限分配,生成页表数据段,
该数据段负责映射虚拟地址和物理地址,设置页访问权限。
其中用户态的代码页和数据页都需要一个额外的内核态页表项进行映射,
确保当程序在 S 模式执行时仍然可以顺利执行。\par

\subsection{策略生成}

为了使得 train group、victim group、decode group 中的各个代码块有较好的指令覆盖率和训练效果,
生成框架对每一个代码块都制定了有针对性的生成策略。本章节将对每个代码块的生成策略进行详细介绍。
为了可以对指令的字段和操作数进行约束求解,本工作扩展开源的 RISCV 指令生成工具razzle\cite{razzle}作为基础指令生成器。

\subsubsection{瞬态窗口触发部分}

瞬态窗口触发部分的重要代码块包括 train group 的训练块(train 块)和trigger group 的触发块(trigger 块)、
延迟块(delay 块)。我们依次对每个块的生成策略进行介绍。\par

\textbf{延迟块(delay块):}\par
delay 块的目的在于让后续的 trigger 块延迟获得操作数结果,迫使 trigger 块预测执行。为了防止延迟块的指令影响前端部件
和内存部件的微架构状态,delay 块主要采用整数运算指令、乘除法扩展指令、单精度浮点指令和双精度浮点指令进行运算。
为了可以让指令的计算尽可能耗时,delay 块应该尽可能多地采用多周期指令,且指令序列之间应该有较强的数据流依赖,
以确保指令的计算开销可以不断累加起来。\par

为了检验哪些类型的指令组合可以产生较长时间的延迟,进而打开足够大的瞬态窗口,我们进行了如下实验:该实验用浮点指令组合和整型指令组合
充当 delay 块,相邻指令之间有数据依赖,用分支预测指令作为 trigger 块,在瞬态窗口中填充 INFO\_TEXT\_START 指令。然后我们执行该测试程序,
根据 rob 插桩监测到的 INFO\_TEXT\_START 指令条目来估计瞬态窗口的大小。测试结果如表\ref{table:float-seq}、\ref{table:int-seq}所示。\par

评估表\ref{table:float-seq}可以看到,一条浮点指令触发的瞬态窗口长度为 8 条指令,两条浮点指令触发的瞬态窗口长度在 14-22 条指令之间,
三条浮点以上指令触发的瞬态窗口长度在 17-24 条指令之间,最大不超过 24 条指令。
故而浮点指令条数在 3-5 条,即可打开 20-24 大小的瞬态窗口,继续增长指令序列没有意义。\par

\begin{table}[h!]
    \begin{center} 
    \caption{浮点指令组合} 
    \label{table:float-seq}  
    \resizebox{1.0\linewidth}{!}{
        \begin{tabular}{|l|l|l|l|l|l|} 
            \hline
            \textbf{指令类型} & \textbf{扩展类型} & \textbf{窗口宽度} & \textbf{指令类型} & \textbf{扩展类型} & \textbf{窗口宽度}\\
            \hline
            cvt                & 1 F/D & 8  & nmadd fsgnjx fcvt                             & 3 F/D & 21 \\
            nmadd fcvt         & 2 F/D & 14 & nmadd fcvt   fcvt                             & 3 F/D & 21 \\
            madd  fcvt         & 2 F/D & 14 & sgnjx fmin   fmadd  fcvt                      & 4 F/D & 24 \\
            cvt   fcvt         & 2 F/D & 15 & sgnjx fsqrt  fsqrt  fsqrt  fcvt               & 5 F/D & 21 \\
            sub   fcvt         & 2 F/D & 22 & add   fmax   fdiv   fmax   fcvt               & 5 F/D & 24 \\
            sgnjn fmadd  fcvt  & 3 F/D & 17 & div   fdiv   fdiv   fdiv   fdiv   fdiv   fcvt & 7 F/D & 24 \\
            madd  fmadd  fcvt  & 3 F/D & 20 &                                               &       &    \\
            \hline
        \end{tabular}
    }
    \end{center}
\end{table}

评估表\ref{table:int-seq}可以看到,任何指令序列都可以打开大小为 3 的瞬态窗口,这说明 branch 预测本身就会导致 3 个周期的瞬态窗口。
3 条整型指令无法进一步扩大瞬态窗口,1 条乘除法指令和 2 条立即数指令配合可以打开 10-12 大小的瞬态窗口,
2 条乘除法指令和 1 条立即数指令配合可以打开 20-24 大小的瞬态窗口,最大窗口不超过 24 条指令。
可以看到只有 M 扩展的指令可以扩大瞬态窗口,其中 div/rem 指令的效果略好于 mul 指令,乘法指令在 3-5 条大概率可以打开最大瞬态窗口。\par

\begin{table}[h!]
    \begin{center} 
    \caption{整数指令组合} 
    \label{table:int-seq}  
    \resizebox{0.9\linewidth}{!}{
        \begin{tabular}{|l|l|l|l|l|l|} 
            \hline
            \textbf{指令类型} & \textbf{扩展类型} & \textbf{窗口宽度} & \textbf{指令类型} & \textbf{扩展类型} & \textbf{窗口宽度}\\
            \hline
                and    and    and           & 3 I       &  3  & mulu   remu   sltu          & 1 I + 2 M &  20 \\
                srlw   mulhsu addiw         & 2 I + 1 M &  10 & div    srli   sub           & 2 I + 1 I &  21 \\
                mulhsu add    divu          & 2 I + 1 M &  12 & divw   addiw  div           & 1 I + 2 M &  24 \\
                sub    divw   add           & 2 I + 1 M &  14 & mulw   sll    slliw   srli  & 3 I + 1 M &  12 \\
            \hline
        \end{tabular}
    }
    \end{center}
\end{table}

最终我们得到结论:序列长度为 3-5 的 M 型指令和 F 型指令混合序列即可打开大小在 18-24 之间的瞬态窗口。这里预留了两个参数,
一个是指令序列的长度,一个是浮点指令的占比,这两个参数可以由外部的配置生成器来产生。\par

延迟块的生成步骤如下:
\begin{enumerate}
    \item 生成寄存器依赖序列:在 delay 块生成时,根据超参数指令序列长度和浮点指令占比,
随机产生一个数据流依赖序列,然后开始根据这个依赖序列依次进行指令生成。\par

    \item 生成指令序列:之后根据依赖序列依次构造指令。
如果源寄存器和目的寄存器都是浮点寄存器则生成浮点指令;如果都是整数寄存器则生成 M 扩展指令;
如果一个是浮点寄存器、一个是通用寄存器,则生成类型转换指令。\par

    \item 事件管理:在 delay 块的开始和末尾插入 label 指令监控 delay 操作的开始和瞬态窗口的关闭。\par

    \item 操作数调整:因为浮点操作的计算结果很容易是全 0 和全 F 的极端值,
这种值在特殊情况下会导致后续 trigger 指令的指令约束永远无法满足。因此这里再随机产生一条指令,让结果寄存器和一个随机数做加法,
让计算结果几乎 100\% 不为全 0 和全 F。\par
\end{enumerate}

\textbf{触发块(trigger 块):}\par
触发块用于产生触发瞬态窗口的代码,主要是利用这部分代码的预测执行错误来产生错误的执行流。生成框架将瞬态窗口执行的 trigger 指令
分为了 22 类,囊括了 rv64imafdc 的所有指令和 zicsr 扩展的部分指令。
22 种 trigger 指令的类别、含义、需要满足的约束如表\ref{table:trigger-gen}所示。
trigger 指令的种类被作为该块的外部配置被配置生成单元管理。trigger 块的生成包括如下两个部分:\par

\begin{table}[h!]
    \begin{center} 
    \caption{触发块指令生成} 
    \label{table:trigger-gen}  
    \resizebox{1.0\linewidth}{!}{
        \begin{tabular}{|l|l|l|} 
            \hline
            \textbf{指令类型} & \textbf{类别解释} & \textbf{操作数约束}\\
            \hline
            LOAD\_MISALIGN      &   触发load\_misalign 异常           & 内存访问地址不对齐                                         \\
            LOAD\_ACCESS\_FAULT &   触发load\_access\_fault 异常      & 地址对齐且落在 access\_fault\_data 块               \\
            LOAD\_PAGE\_FAULT   &   触发load\_page\_fault 异常        & 内存对齐且落在 page\_fault\_data 块                 \\
            STORE\_MISALIGN     &   触发store\_misalign 异常     & 内存访问地址不对齐                                         \\ 
            STORE\_ACCESS\_FAULT&   触发store\_access\_fault 异常& 地址对齐且落在 access\_fault\_data 块               \\
            STORE\_PAGE\_FAULT  &   触发store\_page\_fault 异常  & 地址对齐且落在 page\_fault\_data 块                 \\
            AMO\_MISALIGN       &   触发amo\_misalign 异常       & 内存访问地址不对齐                                         \\ 
            AMO\_ACCESS\_FAULT  &   触发amo\_access\_fault 异常  & 地址对齐且落在 access\_fault\_data 块               \\
            AMO\_PAGE\_FAULT    &   触发amo\_page\_fault 异常    & 地址对齐且落在 page\_fault\_data 块                 \\
            EBREAK              &   ebreak指令触发ebreak异常                  & \textasciitilde                                             \\
            ILLEGAL             &   illegal指令触发illegal异常                & \textasciitilde                                             \\
            ECALL               &   ecall指令触发ecall异常                    & \textasciitilde                                             \\
            V4                  &   store指令与load指令地址冲突                &  操作数a0为access\_secret的访问地址                      \\
            BRANCH              &   分支预测指令,如beq、bne等                 &  控制流转移到 return 块    \\
            JALR                &   jalr指令                                 &  跳转目标为 return 块                                        \\
            RETURN              &   return指令                               &  跳转目标为 return 块                                        \\
            JMP                 &   jal指令                                  &  跳转目标为 return 块                                        \\
            INT                 &   整数运算指令和对应的压缩指令      & \textasciitilde                                             \\
            FLOAT               &   浮点运算指令和对应的压缩指令    & \textasciitilde                                             \\
            LOAD                &   内存载入指令且不触发异常          & 地址对齐且落在 random\_data 块                 \\ 
            STORE               &   内存存储指令且不触发异常          & 地址对齐且落在 random\_data 块                    \\ 
            AMO                 &   原子指令且不触发异常                       & 地址对齐且落在 random\_data 块                    \\ 
            \hline
        \end{tabular}
    }
    \end{center}
\end{table}

\begin{enumerate}
    \item 构造数据流延迟依赖:为了让 trigger 指令的操作数被延迟获得,生成框架为 trigger 指令的源寄存器和 delay 块的计算结果 delay\_reg 产生数据依赖。
    因为 trigger 的指令操作数需要满足特殊的条件约束,例如访存指令只能访问给定的地址区域,跳转指令只能跳入 return 块等,因此需要对
    trigger 指令操作数的值进行约束。\par
    \item 操作数约束调整:因为操作数依赖的 delay\_reg 值是随机产生的,所以生成框架使用寄存器 a0 存储满足约束的调整值,
    当 a0 和 delay\_reg 相加后即可得到满足约束条件的操作数值。为了防止 a0 寄存器的值被中途改写,delay 块不允许使用 a0 寄存器参与计算。\par
\end{enumerate}

可以参看图\ref{code:trigger-example}的代码样例,这里 t0 寄存器是 delay 块的计算结果。a0 寄存器的参数值是 0,
在执行 add 指令之后 a0 寄存器的值等于 t0 寄存器,且 a0 寄存器的值依赖于 t0 寄存器。之后执行 beq 指令,因为 t0、a0 都依赖于
delay 块,所以 branch 指令会被延迟提交开始分支预测;因为 a0 参数调整使得 t0 等于 a0,所以可以跳转到 return\_block\_entry,
进入 return 块。\par

\begin{figure}[htbp]
    \centering
    \begin{minted}{c}
        add a0, t0, a0
        beq a0, t0, return_block_entry
    \end{minted}
    \caption{trigger样例}
    \label{code:trigger-example}
\end{figure}

\textbf{训练块(train 块):}\par
训练块用于对瞬态窗口进行训练,为了覆盖尽可能多的指令种类,train 块的训练指令被细分为 12 类,囊括 rv64imafdc 的所有指令和 zicsr 的部分指令。
所有 12 类指令的类别、含义、满足的约束如表\ref{table:train-gen}所示。其中内存范围限定在 random\_data 块、page\_fault\_data 块和 
access\_fault\_data 块。\par

\begin{table}[h!]
    \begin{center} 
    \caption{训练块指令生成} 
    \label{table:train-gen}  
    \resizebox{1.0\linewidth}{!}{
        \begin{tabular}{|l|l|l|} 
            \hline
            \textbf{指令类型} & \textbf{类别解释} & \textbf{操作数约束}\\
            \hline
            TAKEN               &   分支预测指令且发生跳转                     &  跳转目标为 return 块或者 nop\_return 块,发生跳转              \\
            NOT\_TAKEN          &   分支预测指令且不发生跳转                   &  跳转目标为 return 块或者 nop\_return 块,不发生跳转            \\
            JALR                &   jalr指令                                 &  跳转目标为 return 块或者 nop\_return 块                                        \\ 
            CALL                &   call指令                                 &  跳转目标为 return 块或者 nop\_return 块                                        \\
            RETURN              &   return指令                               &  跳转目标为 return 块或者 nop\_return 块                                        \\
            JMP                 &   jal指令                                  &  跳转目标为 return 块或者 nop\_return 块                                        \\
            INT                 &   整数运算指令和对应的压缩指令      & \textasciitilde                                             \\
            FLOAT               &   浮点运算指令和对应的压缩指令    & \textasciitilde                                             \\
            LOAD                &   内存载入指令,且不触发异常          & 访问内存地址落在指定范围内\\
            STORE               &   内存存储指令,且不触发异常          & 访问内存地址落在指定范围内\\
            AMO                 &   原子指令且不触发异常                       & 访问内存地址落在指定范围内\\
            SYSTEM              &   系统指令               & \textasciitilde                                             \\ 
            \hline
        \end{tabular}
    }
    \end{center}
\end{table}

train 块根据是否和 trigger 对齐、是否是单条指令共有三种变体。变体一为指令对齐且单条指令,该类型随机生成某一类指令,然后将该指令和 trigger 指令对齐,
寄存器的值由 load\_init\_train 提供,确保操作数满足指令约束。变体二为指令不对齐且单条指令,该类型随机生成某一类指令,
但是排布位置在 trigger 指令地址的基础上随机偏移一段距离。变体三为多条指令,
该变体使用 arbitrary 块生成。\par

\textbf{随机块(arbitrary 块):}\par
为了提高测试程序的指令组合覆盖率,生成框架需要一些纯随机的指令块进行测试。其生成过程包括如下两个步骤:
\begin{enumerate}
    \item 随机指令生成:生成框架提供了五类随机指令生成函数(函数功能参见表\ref{table:random-gen}),用于生成整型、浮点、内存访问、原子操作、系统调用指令。
为了防止修改特权寄存器导致测试程序无法正常工作,这里不生成 CSR 读写指令。\par
    \item 指令块生成和组合:arbitrary 块参考 razzle 的随机代码生成方式,首先随机生成若干个基本块,每个基本块随机生成若干条非控制流指令,
    然后将这些基本块之间用 branch 和 jmp 等控制流指令连接起来,确保基本块控制流可以从第一个基本块进入,最后一个基本块退出(触发异常退出除外)。
\end{enumerate}

\begin{table}[h!]
    \begin{center} 
    \caption{随机指令生成} 
    \label{table:random-gen}  
    \resizebox{0.8\linewidth}{!}{
        \begin{tabular}{|l|l|l|} 
            \hline
            \textbf{函数名} & \textbf{生成指令类别} & \textbf{额外约束}\\
            \hline
            \_gen\_atomic             &  原子操作指令             & 访问内存地址落在指定范围内\\
            \_gen\_float\_arithmetic   &  浮点运算指令及其压缩指令  &  \textasciitilde  \\
            \_gen\_int\_arithmetic     &  整数运算指令及其压缩指令  &  \textasciitilde  \\
            \_gen\_load\_store         &  内存访问指令及其压缩指令  & 访问内存地址落在指定范围内\\
            \_gen\_system              &  系统调用指令             &  \textasciitilde  \\
            \hline
        \end{tabular}
    }
    \end{center}
\end{table}

对于 train 块的变体 3,我们可以使用 arbitrary 块产生多条随机的训练指令,用于 trigger 块的训练;arbitrary 块也可以被用于其他需要随机指令生成的场景。\par

\subsubsection{寄存器初始化部分}

load\_init 块和 load\_init\_train 块负责为 group 中所有没有被初始化的指令提供初始化的值。
例如 delay 块的随机指令中的寄存器很多是没有赋初值的,
需要 load\_init 块提供随机值;trigger 块和 train 块满足约束的寄存器值也要由 load\_init 计算得到。
其中 victim group和 decode group 的代码初始化较为复杂,用 load\_init 块进行初始化;
而 train group 的初始化较为简单,用 load\_init\_train 块初始化。
下面我们针对 load\_init 块的生成策略进行介绍,load\_init\_train 块的生成策略基本一致,便不予过多赘述。\par

\begin{algorithm}[!h]
    
    \caption{基本块寄存器分析}
    \label{alg:basic-reg-analysis}
    \renewcommand{\algorithmicrequire}{\textbf{Input:}}
    \renewcommand{\algorithmicensure}{\textbf{Output:}}
    
    \begin{algorithmic}[1]
        \REQUIRE basic block list of code block, block\_list  %%input
        \ENSURE  basic block list of code block's register initialized result, block\_list  %%output
        \FOR{$block \in block\_list$}
            \STATE $block.need\_init\_set \Leftarrow \emptyset$
            \STATE $block.has\_inited\_set \Leftarrow \emptyset$
            \FOR{$inst \in block$}
                \FOR{$dest in inst$}
                    \STATE $block.has\_inited\_set \Leftarrow block.has\_inited\_set \bigcup dest$
                \ENDFOR
                \FOR{$src \in inst$}
                    \IF{$src \notin block.has\_inited\_set$}
                        \STATE $block.need\_init\_set \Leftarrow block.need\_init\_set \bigcup src$
                    \ENDIF
                \ENDFOR
            \ENDFOR
            \STATE $block.has\_inited\_set \Leftarrow block.has\_inited\_set \bigcup block.need\_inited\_set$
        \ENDFOR
        \RETURN block\_list
    \end{algorithmic}
\end{algorithm}

\textbf{未初始化寄存器收集:}
为了给没有初始化的寄存器赋初值,load\_init块首先需要收集 group 的所有块中没有被初始化的寄存器。
group 的每个代码块都由一系列基本块组成,
所以生成框架首先会使用算法\ref{alg:basic-reg-analysis}计算这些基本块各自需要被额外初始化的寄存器。
该算法遍历所有指令,将目的寄存器加入 has\_inited\_set 集合,将没有被初始化的源寄存器加入 need\_init\_set 集合。
这两个集合将被用于后续计算。
因为 need\_init\_set 被要求在基本块被执行之前被初始化,所以 has\_inited\_set 额外包含  need\_init\_set。 \par

\begin{algorithm}[!h]
    
    \caption{代码块寄存器分析}
    \label{alg:block-reg-analysis}
    \renewcommand{\algorithmicrequire}{\textbf{Input:}}
    \renewcommand{\algorithmicensure}{\textbf{Output:}}
    
    \begin{algorithmic}[1]
        \REQUIRE basic block list of code block, block\_list    %%input
        \ENSURE  the set of the register need to be initialized, need\_inited\_set  %%output
        \STATE $need\_inited\_set \Leftarrow \emptyset$
        \WHILE{$\bigcup block.need\_inited\_set \succ need\_inited\_set$}
            \STATE $need\_inited\_set \Leftarrow \bigcup block.need\_inited\_set$
            \FOR{$block \in block\_list$}
                \STATE $block.need\_inited\_set \Leftarrow block.need\_inited\_set - \bigcap block.previous.has\_inited\_set$
                \STATE $block.has\_inited\_set \Leftarrow block.has\_inited\_set - \bigcap block.previous.has\_inited\_set$
            \ENDFOR
        \ENDWHILE
        \RETURN need\_inited\_set
    \end{algorithmic}
\end{algorithm}

因为在代码块内部,如果基本块 1 在基本块 0 之后被执行,而基本块 0 初始化了基本块 1 的待初始化寄存器,
则这些寄存器不需要被 load\_init 块额外初始化。因此生成框架还需要根据基本块之间的控制流关系,进一步细化真正需要被初始化的寄存器。
为了表述基本块之间的控制流关系,每个基本块的 previous 和 succeed 变量记录能进入这个块的基本块,和这个基本块可以进入的基本块。
使用一个迭代算法,计算每个基本块的前继块共同初始化的寄存器集合,然后该基本块的 has\_inited\_set 集合加入这部分寄存器,
need\_inited\_set 集合减去这部分寄存器。具体细化方法详见算法\ref{alg:block-reg-analysis}。\par

因为代码块之间也存在基本块之间的寄存器初始化传递关系,所以以代码块为处理对象再次执行算法\ref{alg:block-reg-analysis},
即可收集到所有要初始化的寄存器组。\par

\textbf{寄存器随机初始化:}
对于已经得到的待初始化寄存器序列,生成器产生一一对应的初始化随机数和初始化代码。为了使初始化代码更加简短,这里使用压缩指令
c.fldsp 和 c.ldsp 指令初始化浮点寄存器和通用寄存器,sp指令充当随机数序列的基地址,立即数充当序列的偏移量。考虑到 a0 寄存器被用于
传递 trigger 块所需的特殊参数,所以 a0 块在最后被初始化。因为 sp 寄存器被用于作为内存访问的基地址,如果 sp 寄存器也需要被随机数初始化,
则将 sp 寄存器放在初始化序列的末尾进行初始化。\par

\textbf{寄存器约束值初始化:}
对寄存器值有约束需求的寄存器,生成框架需要额外计算这部分寄存器的初始值,而不是简单随机生成。
对于 load\_init\_train 块的场景可以直接根据 train 块的约束计算得到 train 指令需要的值。但是对于 load\_init 块的场景,
要求 a0 寄存器的值和 delay 块的计算结果相加之后满足约束,所以需要事先知道 delay 块的执行结果。\par

我们对开源 RISCV 模拟器 spike 进行扩展,编写了 spike-solve 程序。该程序会载入待模拟的 RISCV elf 文件,然后模拟执行指定的
前 N 条指令,并将模拟得到的寄存器值 dump 出来。为了得到 delay 块的执行结果,生成框架将 load\_init 块和 delay 块的
指令和数据组合为简单的可执行程序,然后交给 spike-solve 模拟求解,最终得到 delay 块的执行结果。因为上述代码中唯一待定的 a0 寄存器的值
不参与 delay 块的计算,所以不影响最终的计算结果。之后生成框架根据 trigger 块的指令约束和 delay 块的寄存器结果求解 a0 的参数值即可。

\subsubsection{秘密数据访问部分}

秘密数据访问涉及到 victim group 的秘密数据访问块( access\_secret 块)和秘密数据迁移块(secret\_migrate 块),
前者负责将存储结构中的秘密数据载入寄存器,
后者负责将秘密数据迁移到合适的储存深度。\par

\textbf{秘密访问块(access\_secret 块):}\par
access\_secret 块用于访问 secret 地址,将 secret 数据载入到寄存器中。这一步归根结底就是生成一个内存地址,然后进行内存的 load 操作,
所以为了简单起见,本生成框架并不用随机生成指令来构造内存地址,而是直接先验地生成内存地址,然后使用该地址进行 load 操作。\par

access\_secret 块有两部分代码组成:
\begin{enumerate}
    \item 地址构造:生成的内存地址可以直接等于 secret 的地址;但是考虑到一些内存访问场景,
缓冲区只需要低位匹配就可以获得缓冲区缓存的数据。为了覆盖该种类型的漏洞,
生成框架在构造地址时会将地址的高位随机替换为一个随机数,被替换的高位在 0 位到 28 位的范围中随机选择。\par
    \item 地址获取:load 指令的地址寄存器有两种方法获得这个预设地址。一是使用立即数指令直接构造获得;
二是使用内存载入指令访问 data\_victim 块获得。在构造 group 的时候,因为生成器需要指令的地址满足 group 间的对齐,
所以在生成指令的时候就需要知道指令的长度,但是立即数构造指令 li 指令的真实指令长度是随着立即数的复杂程度变化的,
因此地址被随机数高位覆盖时,只能使用方法二获得地址。
此外,spectre-V4 类型的瞬态执行漏洞通过覆盖access\_secret 块存储的 secret 地址偏移量来触发瞬态窗口,
所以如果 trigger 类型为 V4,则只能使用方法二获得地址。\par
\end{enumerate}

地址的构造方式和获取方式可以作为外部配置,由外部配置生成器产生。\par

\textbf{秘密迁移块(secret\_migrate 块):}\par
secret\_migrate 块用于将 secret 从 memory 转移到内存存储层次的其他存储位置,
以检验 access\_secret 块对不同存储层次中 secret 访问成功的可能性。\par

在 boom 处理器的环境上,secret\_migrate 块支持三种存储位置的迁移,依次介绍如下。
secret 的迁移被作为该块的外部配置,被配置生成单元管理:
\begin{enumerate}
    \item 内存 memory:因为 secret 本来就在 memory 中,所以无需指令做额外调整。\par
    \item 缓存 cache:用立即数指令生成 secret 地址,然后用 load 指令显式地将 secret 载入到寄存器,
然后将寄存器填零,即可将 secret 移入 cache 但不移入寄存器。
因为差分测试的时候本体和变体都会执行该操作,所以虽然访问了 secret 但是不会造成执行时间的差异。\par
    \item 线性载入缓冲区 line fill buffer:数据从 memory 被载入到 cache 的过程中,
    一个 cacheline 的数据首先会被缓存到 line fill buffer,再写入 cacheline。
    为了让 secret 进入 line fill buffer 但是不进入 cache,生成框架首先显式载入 secret,
    让 secret 进入 line fill buffer 和 cache,然后对 dummy\_data 四个页中和 cacheline 地址冲突的地址进行写操作。
    因为 boom 的 cache 是四路组关联、伪随机替换的,所以做四次写操作大概率可以将 cache 中的 secret 覆盖。
    为了防止后续的 load 操作覆盖 line fill buffer 中的 secret,secret\_migrate 块被排布在 delay 块、trigger 块之前,
    之后紧接着执行 access\_secret 块,尽量保证在执行 access\_secret 块前没有内存访问操作。\par
\end{enumerate}

为了防止 secret\_migrate 块的寄存器修改影响 delay 块、load\_init 块等代码块的寄存器结果,
secret\_migrate 块在其他所有块都生成完毕后生成。然后 secret\_migrate 块选择没有被 load\_init 块初始化的寄存器作为内部使用的寄存器,
这样就可以避免 secret\_migrate 块修改的寄存器值被后续的代码块使用。另外,三种类型的 secret\_migrate 块长度是互不相同的,
为了方便程序仅仅替换 secret\_migrate 块的迁移类型时不会影响其他代码块的地址布局,这里还需要额外补充 nop 指令,确保三种类型的 secret\_migrate 块长度相同。\par

\subsubsection{秘密数据泄露部分}

秘密数据泄露代码块包括用于秘密数据编码的 victim group 的编码块(encode 块)
和用于秘密数据解码的 decode group 的解码块(decode 块)。
其中encode 块是阶段2的核心代码块,负责在瞬态窗口中将秘密数据编码到微架构状态当中。
encode 块需要尝试在瞬态窗口内尽可能多地使用携带秘密数据的寄存器(我们称为 secret reg),
然后尝试组合各种类型的指令,尝试将 secret 编码到各种类型的微架构部件中去。\par

\textbf{控制流类型和数据流类型:}
秘密数据既可以通过 add、ld、fmul 等指令影响数据流及数据流部件,也可以通过 branch、jmp 等指令影响控制流及控制流部件。
对于一个包含了 secret reg 的基本块,该基本块在执行的时候就会让 secret 影响微架构部件的状态,进而尝试在微架构中编码 secret,这就是数据流类型。
对于不包含 secret reg 的基本块,为了让该基本块的执行也可以受到 secret 的影响,可以将这个基本块包裹在一个涉及 secret reg 的分支语句当中,
则该基本块会根据 secret 值的不同而选择被执行或者不执行,进而影响前端的控制流部件和流水线中的运算、存储部件,这就是控制流类型。
如果一个基本块包含 secret reg,则生成框架将它建模为数据流类型;如果不包含 secret reg 则将它建模为控制流类型。
不过为了简化编码块的生成,生成框架会事先选择生成数据流类型还是控制流类型。
如果是数据流类型,则仅生成与 secret reg 相关的数据流指令,不生成与 secret reg 相关的控制流指令;
如果是控制流类型,则在全局生成一个 secret reg 相关的 branch 分支,确保本体和变体可以走不同的分支,
然后在一个分支中生成不带 secret reg 的指令块,另一个分支保持空
(防止本体和变体编码不同的组件,导致差分测试的时间差异难以评估)。\par

在构造基本块的时候,生成框架希望基本块产生的执行行为尽可能极端,可以让流水线的执行行为尽可能复杂。
所以每个基本块应该尽可能集中地执行单一指令,从而让该类指令涉及到的微架构部件承受足够大的压力,
进而在一些边界条件触发瞬态漏洞,错误地编码 secret 信息。生成框架根据指令类型将基本块分为了
整数操作、浮点操作、内存访问、原子操作、系统调用、函数调用返回、分支预测、跳转、CSR操作九大类。
为了让流水线的执行行为尽可能复杂,基本块的指令之间最好有较强的竞争依赖,比如数据竞争和结构竞争。
最后为了让 secret 信息尽可能多地影响处理器,指令应该尽量使用 secret reg 作为操作数,
并将非 secret reg 作为目的寄存器,在尽可能多地使用 secret 的同时,尽可能多地传播 secret。
delay 块一节的实验表明,瞬态窗口的大小一般在 24 条指令以内,
因此每个基本块长度不超过 6 条指令,总的基本块个数不超过 6 个。\par

\textbf{secret 传播:}
为了让 secret reg 尽可能地被指令使用,也为了让 secret 尽可能多地传播,
生成框架在构造基本块的时候会维护一个 secret reg 集合和一个 normal reg 集合。
之后在依次生成基本块的指令时,以较大概率选用 secret reg 集合的寄存器作为指令的源寄存器,
以较大概率选用 normal reg 集合的寄存器作为指令的目的寄存器。
指令生成完毕之后,根据目的寄存器是否获得或者失去了 secret 信息,
对 normal reg 和 secret reg 集合进行更新调整,确保集合信息正确。
此外,因为指令尽可能多地使用了 secret reg,而 secret reg 中的指令都是随着 access\_secret 产生的 secret reg
通过数据流依赖传播过来的,所以这些指令之间很容易产生数据流依赖,可以同时满足指令间复杂依赖的需求。\par

\textbf{整数操作块、浮点操作块、系统调用块、CSR操作块:}
这四个块分别生成一系列整数指令、浮点指令、系统调用指令、CSR操作指令。
在生成指令的寄存器字段的时候,采用 secret 传播的策略,提高 secret 的使用率和指令间的依赖程度。
为了确保指令的长度是可以在生成汇编时确定的,操作块不生成 la、li 等伪指令。
在 spectre-V4 情况下,encode 块的指令会被执行,此时不允许生成 CSR 块,防止特权寄存器的修改破坏程序的执行。\par

\textbf{内存访问块和原子操作块:}
内存访问块和原子操作块都涉及内存操作,因此需要获得合法的内存访问地址。
首先产生 la 指令,确保获得的基地址在 channel 块、random\_data 块内。
之后为了让地址可以随机一些,也为了让访问的地址可以受到 secret 的影响,生成框架按概率选择一个寄存器,
将寄存器的值和 0x7f8 计算,然后和 la 得到的基地址相加,
然后以该地址为基地址生成多条内存访问指令或者原子操作指令即可。\par

\textbf{分支预测块、函数调用返回块、跳转块:}
分支预测块由一系列 branch 块组成,主要影响 BIM(bi-modal table)部件;函数调用返回块由一系列 call-return 指令组成,主要影响 RSB 部件(return stack buffer);
跳转块由一系列 jalr、jal 指令组成,主要影响 BTB 部件(branch target buffer)。因为跳转的目标对于部件的训练本身意义不大,
同时也为了可以让控制流块的指令都可以被执行,这里每条控制流指令的跳转目标都是下一条指令。
因为三个块基本不涉及 secret reg 的使用,因此只能被放在控制流类型的场景中。\par

\textbf{块的组合:}
encode 的块序列由一组基本块序列和结尾的 nop 填充块组成,这些块依次执行。如果 encode 块是控制流类型,则在基本块中间插入一个 secret reg 相关的 branch 指令,
让变体跳转到 nop 块执行空操作,而让本体继续执行有意义的指令,如图\ref{paper:encode-consist}的V3、V4。如果 trigger 类型是 spectre-V4 类型,
encode 的代码会被直接执行,所以生成框架让基本块都保持顺序执行,防止出现死循环无法停机,
如图\ref{paper:encode-consist}的V2、V4;但是对于其他的 trigger 类型,encode 的基本块不会被正式执行,
所以可以让最后一个基本块无条件跳转到第一个基本块,在瞬态窗口中充分执行 encode 基本块的指令,如图\ref{paper:encode-consist}的V1、V3。\par

\begin{figure}[!h]
    \centering
    \includegraphics[width=\linewidth]{figure/paper/encode-consist.png}
    \caption{基本块控制流}
    \label{paper:encode-consist}
\end{figure}

\textbf{解码块(decode 块):}\par
decode 块和 encode 块配合使用,用于将编码在微架构状态中的 secret 利用时间侧信道解码出来。
考虑到在同样的地址执行同样的指令往往可以使用一样的微架构组件参与执行,
因此生成器简单地将 encode 块共用地址且内容相同的代码块作为 decode 块。
需要注意的是,encode 块内部可能存在死循环,因此 decode 块需要将这个死循环打开。
此外 decode 块用 INFO\_LEAK\_START 和 INFO\_LEAK\_END 作为
decode 块的开始和结束,用于表示侧信道泄露秘密数据这一过程的开始和结束。\par

\textbf{替换块(replace 块):}\par
replace 块在 decode group 中被使用,用于替换 victim group 中 trigger 块、access\_secret 块的内容,
如果 trigger 块和 access\_secret 块之间存在额外的 return 块,则需要将 return 块也替换掉。
因为 trigger 块和 return 块在这里不需要被使用,所以生成框架将 replace 块中的对应区域简单地替换为 nop 指令,
但是对于 access\_secret 块,后续的 decode 块需要从中得到和 secret 值相同的操作数,以此来复现 encode 块的指令执行,
所以 access\_secret 块的部分被替换为立即数构造 secret 值的简单代码。
因为 load\_init 块的指令和数据可能携带了 return 块、access\_secret 块的 label,所以在生成 replace 块的时候
需要保留它们的 label 信息,且这些 label 要和 victim group 的 label 对齐,以确保 load\_init 块的数据保持不变。\par

\cleardoublepage
\section{评估及结果分析}

在本章节,我们对瞬态漏洞测试程序生成框架进行评估,以体现我们测试程序生成框架的高效性和高覆盖率。
对于瞬态窗口触发部分,我们对 train 块的三种变体进行测试,通过比较,体现我们单条指令-地址敏感设计的有效性,
以及瞬态窗口触发类型挖掘的高覆盖率。随后我们对整体的瞬态漏洞测试程序生成进行评估,
通过统计挖掘完整瞬态漏洞 PoC 的时间开销和类型覆盖率,验证我们设计的高效性。
最后我们对特定的瞬态漏洞侧信道进行定向挖掘,通过和 SpecDoctor 的挖掘时间进行横向对比,验证我们设计的高效性。\par

\subsection{评估环境和测试框架搭建}

为了进行 RISCV 处理器的仿真和测试,我们使用硬件扩展后的开源处理器评估框架 starship 作为测试平台,
使用 boom 处理器作为瞬态漏洞挖掘的测试对象,使用 vcs 作为处理器仿真执行的综合工具。
本实验在 16 物理核,3.4GHz主频,128G内存的 AMD 5950X 服务器上进行测评。\par

测试框架由两阶段组成。阶段 1 如图\ref{paper:stage1-test}所示,首先随机产生一个测试框架代码块配置,
然后生成 train group、trigger group 的代码,执行 starship 平台的
boom 处理器仿真测试,如果检测到瞬态窗口触发就保存这个 trigger 样例,
进入下一阶段,不然就产生一组新的配置,继续生成新的测试代码。
每轮迭代的配置和生成结果都会被暂存起来,供后续数据分析。\par

\begin{figure}[!h]
    \centering
    \includegraphics[width=\linewidth]{figure/paper/stage1-test.png}
    \caption{阶段1测试}
    \label{paper:stage1-test}
\end{figure}

阶段 2 如图\ref{paper:stage2-test}所示,在阶段 1 程序的基础上产生阶段2的代码块配置,
继而生成后续的 trans 部分和 decode group,然后在 starship 平台执行差分测试,
如果检测到执行时间不一致就保存这个 leak 样例。
每轮迭代的配置和生成结果都会被暂存起来,供后续数据分析。\par

\begin{figure}[!h]
    \centering
    \includegraphics[width=\linewidth]{figure/paper/stage2-test.png}
    \caption{阶段2测试}
    \label{paper:stage2-test}
\end{figure}

\subsection{触发瞬态窗口效率评估}

为了检验生成框架使用单条指令-地址敏感的训练代码生成策略的有效性,我们分别使用 train 块的变体 V1(单条指令-地址对齐)、
V2(单条指令-地址不对齐)、V3(多条指令)生成 train 部分的代码,然后分别用 stage 1 测试框架进行 1 小时的测试,
并对瞬态窗口触发程序生成的轮数、找到的瞬态窗口数、瞬态窗口类型占比等进行统计。\par

\subsubsection{测试结果分析}

V1、V2、V3进行一个小时测试之后,我们对各个变体找到的瞬态窗口类型进行统计。经过统计我们发现,train group 规约之后,trigger 指令
最多经过一个 train group 的训练就可以触发瞬态窗口,因此我们使用 trigger 指令的类型和训练 trigger 指令的
train 指令的类型组合来表征触发瞬态窗口的类别。表\ref{table:v1-trans}、\ref{table:v2-trans}、\ref{table:v3-trans},
依次展示了 V1、V2、V3 找到的瞬态窗口种类、该类瞬态窗口的个数、该类瞬态窗口在大类中占比等信息。
后续我们将通过对 V1、V2、V3 的实验数据进行分析比较,验证我们训练策略的有效性。\par

V1 在单条指令、地址对齐的情况下测试一个小时,共进行了 447 轮迭代,找到 228 个瞬态窗口。其中 jalr 跳转预测错误类型 5 类 67 个,
spectre-V4 存储预测错误类型 2 类 39 个,return 返回地址预测错误类型 2 类 35 个,branch 分支预测错误类型 3 类 30 个,
meltdown 异常延迟检测类型 6 类 57 个。其中通过训练触发的瞬态窗口共计 106 个。\par

\begin{table}[h!]
    \begin{center} 
    \caption{V1 瞬态窗口统计} 
    \label{table:v1-trans}  
    \resizebox{1.0\linewidth}{!}{
        \begin{tabular}{|l|l|l|l|l|l|l|l|} 
            \hline
            \textbf{trigger 类别} & \textbf{train 类别} & \textbf{窗口数} & \textbf{大类占比} & \textbf{trigger 类别} & \textbf{train 类别} & \textbf{窗口数} & \textbf{大类占比}\\
            \hline
            JALR                &TAKEN              &4  &5.97\%   &BRANCH              &NOT\_TAKEN         &11 &36.67\%  \\
            JALR                &JALR               &33 &49.25\%  &BRANCH              &NONE               &15 &50.00\%  \\
            JALR                &RETURN             &14 &20.90\%  &BRANCH              &TAKEN              &4  &13.33\%  \\
            JALR                &CALL               &13 &19.40\%  &STORE\_ACCESS\_FAULT&NONE               &10 &100.00\% \\
            JALR                &JMP                &3  &4.48\%   &STORE\_PAGE\_FAULT  &NONE               &12 &100.00\% \\
            V4                  &NONE               &34 &87.18\%  &STORE\_MISALIGN     &NONE               &3  &100.00\% \\
            V4                  &AMO                &5  &12.82\%  &LOAD\_ACCESS\_FAULT &NONE               &15 &100.00\% \\
            RETURN              &CALL               &31 &88.57\%  &LOAD\_PAGE\_FAULT   &NONE               &15 &100.00\% \\
            RETURN              &JALR               &4  &11.43\%  &LOAD\_MISALIGN      &NONE               &2  &100.00\% \\
            \hline
        \end{tabular}
    }
    \end{center}
\end{table}

V2 在单条指令、地址可以不对齐的情况下测试 1 个小时,共进行了 453 轮迭代,找到 134 个瞬态窗口。其中 jalr 跳转预测错误类型 3 类 8 个,
spectre-V4 存储预测错误类型 4 类 37 个,branch 分支预测错误类型 1 类 26 个,
meltdown 异常延迟检测类型 6 类 63 个。其中通过训练触发的瞬态窗口共计 12 个。\par

\begin{table}[h!]
    \begin{center} 
    \caption{V2 瞬态窗口统计} 
    \label{table:v2-trans}  
    \resizebox{1.0\linewidth}{!}{
        \begin{tabular}{|l|l|l|l|l|l|l|l|} 
            \hline
            \textbf{trigger 类别} & \textbf{train 类别} & \textbf{窗口数} & \textbf{大类占比} & \textbf{trigger 类别} & \textbf{train 类别} & \textbf{窗口数} & \textbf{大类占比}\\
            \hline
            JALR                    &JALR     &3  &37.50\%   &BRANCH                  &NONE     &26 &100.00\%  \\
            JALR                    &RETURN   &4  &50.00\%   &LOAD\_MISALIGN          &NONE     &7  &100.00\%  \\
            JALR                    &CALL     &1  &12.50\%   &LOAD\_PAGE\_FAULT       &NONE     &12 &100.00\%  \\
            V4                      &NONE     &33 &89.19\%   &LOAD\_ACCESS\_FAULT     &NONE     &11 &100.00\%  \\
            V4                      &RETURN   &1  &2.70\%    &STORE\_MISALIGN         &NONE     &5  &100.00\%  \\
            V4                      &AMO      &2  &5.41\%    &STORE\_PAGE\_FAULT      &NONE     &12 &100.00\%  \\
            V4                      &STORE    &1  &2.70\%    &STORE\_ACCESS\_FAULT    &NONE     &16 &100.00\%  \\
            \hline
        \end{tabular}
    }
    \end{center}
\end{table}

V3 在多条随机指令的情况下测试一个小时,共进行了 411 轮迭代,找到 120 个瞬态窗口。其中 jalr 跳转预测错误类型 1 类 2 个,
spectre-V4 存储预测错误类型 1 类 32 个,branch 分支预测错误类型 1 类 34 个,return 返回地址预测错误类型 1 类 1 个,
meltdown 异常延迟检测类型 6 类 51 个。其中通过训练触发的瞬态窗口共计 0 个。\par

\begin{table}[h!]
    \begin{center} 
    \caption{V3 瞬态窗口统计} 
    \label{table:v3-trans}  
    \resizebox{1.0\linewidth}{!}{
        \begin{tabular}{|l|l|l|l|l|l|l|l|} 
            \hline
            \textbf{trigger 类别} & \textbf{train 类别} & \textbf{窗口数} & \textbf{大类占比} & \textbf{trigger 类别} & \textbf{train 类别} & \textbf{窗口数} & \textbf{大类占比}\\
            \hline
            LOAD\_MISALIGN        &NONE &4  &100.00\% &            RETURN              &NONE &1  &100.00\%\\
            LOAD\_PAGE\_FAULT     &NONE &10 &100.00\% &            JALR                &NONE &2  &100.00\%\\
            LOAD\_ACCESS\_FAULT   &NONE &11 &100.00\% &            BRANCH              &NONE &34 &100.00\%\\
            STORE\_MISALIGN       &NONE &6  &100.00\% &            V4                  &NONE &32 &100.00\%\\
            STORE\_PAGE\_FAULT    &NONE &10 &100.00\% & & & & \\
            STORE\_ACCESS\_FAULT  &NONE &10 &100.00\% & & & & \\
            \hline
        \end{tabular}
    }
    \end{center}
\end{table}

\subsubsection{训练块生成策略对比}

\textbf{V1 与 V2 的对比:}
这里将单条指令对齐的训练块生成策略 V1 和单条指令不对齐的策略 V2 的测试结果进行对比,
以此体现训练指令地址敏感对于训练能力提高的重要性。
可以看到对于不需要训练的瞬态窗口触发类型,V1 和 V2 能找到相近个数的瞬态窗口,训练能力保持一致。
但是对于需要训练的瞬态窗口,V1 类型可以找到 10 种、共计 106 个瞬态窗口,
远远高于 V2 的 6 种、共计 12 个瞬态窗口。可以看到控制流类型的瞬态窗口触发指令,
如分支预测错误、跳转目标预测错误、返回地址预测错误类型的训练是地址敏感的,
有意识地让训练块进行地址对齐可以大大提高 trigger 指令训练的成功率。但我们也可以看到 V2 找到了
两种 V1 没有找到的训练类型,即用 return 指令和 store 指令对 spectre-V4 进行训练,也许说明不对齐的 V2 变体
有时可以覆盖一些极其罕见的训练情况,故而 V2 变体在瞬态漏洞挖掘时可以作为 V1 的补充。\par

\textbf{V2 与 V3 的对比:}
这里将单条指令不对齐的训练块生成策略 V2 和 多条指令的训练块生成策略 V3 的测试结果进行对比,
以此说明训练指令条数的增加对于训练能力的提高意义不大,进而为单条指令训练的有效性提供依据。
可以看到 V3 类型因为产生的指令条目更多,训练代码更复杂,所以产生的测试程序更少,比 V2 少了 42 次迭代。
而无论是触发的瞬态窗口的种类,还是触发的瞬态窗口在迭代中的占比,V2 都和 V3 相近,甚至在这次测试中 V3 略低于 V2。
所以在多数情况下 V3 生成更多的训练指令并不能带来瞬态窗口触发上的收益,反而会降低测试程序生成的效率。
因此 V3 应该仅作为 V1 和 V2 的补充,而不能作为主要的生成策略。\par

\subsection{瞬态窗口类型介绍}

V1、V2、V3 收集到的瞬态窗口类型如表\ref{table:trans-summary}所示,这里我们对每种类型的原理进行简单介绍。\par

\textbf{JALR+NONE}的组合利用 jalr 指令的默认跳转目标预测。当 jalr 指令没有被训练时,
它在分支预测时查询对应的 BTB 表,如果发现对应的 BTB 表项正好为空,则默认选择跳转到 PC+4 的地址,
顺序执行下一条指令打开瞬态窗口。\par

\textbf{JALR+RETURN/CALL/JALR}的三种组合本质上是一样的,都是用 jalr 寄存器跳转指令训练寄存器跳转指令。
jalr指令在执行之后会将跳转的目标地址保存到处理器前端的 BTB(branch target buffer)组件中,
然后在后续的 jalr 指令预测执行时会根据查找对应的 BTB 表项得到预测的跳转地址,从而错误地进入瞬态窗口。
该类型就是在训练时用 jalr 指令跳转进入瞬态窗口地址,将瞬态窗口地址写入 BTB,
使得 trigger 块的 jalr 错误地预测进入瞬态窗口。\par

\begin{table}[h!]
    \begin{center} 
    \caption{瞬态窗口统计} 
    \label{table:trans-summary}  
    \resizebox{0.8\linewidth}{!}{
        \begin{tabular}{|l|l|l|} 
            \hline
            \textbf{瞬态窗口类别} & \textbf{预测部件} & \textbf{原理}\\
            \hline
            JALR+NONE               &   BTB &   BTB 默认预测错误      \\
            JALR+CALL/RETURN/JALR   &   BTB &   jalr训练BTB导致预测错误 \\
            JALR+TAKEN/NOT\_TAKEN/JMP&   BTB &   jmp/branch训练BTB导致预测错误 \\
            RETURN+CALL             &   RSB &   call训练RSB导致预测错误 \\
            RETURN+JALR             &   BTB &   jalr训练BTB导致预测错误 \\
            BRANCH+NONE             &   BIM &   BIM初始化值预测错误 \\
            BRANCH+BRANCH           &   BIM &   branch训练BIM导致预测错误 \\
            STORE\_PAGE\_FAULT      &   无  &   异常、权限延迟检测  \\
            STORE\_ACCESS\_FAULT    &   无  &   异常、权限延迟检测  \\
            STORE\_MISALIGN         &   无  &   异常、权限延迟检测  \\
            LOAD\_PAGE\_FAULT       &   无  &   异常、权限延迟检测  \\
            LOAD\_ACCESS\_FAULT     &   无  &   异常、权限延迟检测  \\
            LOAD\_MISALIGN          &   无  &   异常、权限延迟检测  \\
            V4                      &   lsu &   内存未修改预测错误  \\
            V4+AMO/STORE/return     &   lsu &   未知               \\
            \hline
        \end{tabular}
    }
    \end{center}
\end{table}

\textbf{JALR+JMP/TAKEN/NOT\_TAKEN}的组合也是用 jmp 指令和 branch 指令训练了 jalr 指令的 BTB 表项,
但是 jmp 指令和 branch 指令的跳转地址本身其实不需要被预测,它们的跳转目标并不应该
修改 BTB 表项,这侧面反映了 boom 的 BTB 更新策略存在过度更新的漏洞。\par

\textbf{RETURN+CALL}的组合通过训练处理器的 RSB(return stack buffer)触发瞬态窗口。
call 指令执行时会将 pc+4 缓存到 RSB 中作为之后 return 指令的返回地址。我们只需要让
call 指令的下一条指令的地址是瞬态窗口的入口,return 指令预测执行的时候就会进入瞬态窗口。\par

\textbf{RETURN+JALR}的组合利用了 return 指令本身也是 jalr 指令的特性进行工作。这说明了 return 指令
并不仅仅根据 RSB 进行跳转地址的预测,在条件合适的情况下也可以使用 BTB 中的数据进行分支预测。\par

\textbf{BRANCH+TAKEN/NOT\_TAKEN}的组合利用了 branch 指令对 branch 指令的训练。
当 branch 指令执行完毕后,它的跳转与否会被提交到 BIM 组件中,之后 branch 再次执行时会检查 BIM 部件
对应表项的预测结果,然后进行跳转。如果需要 branch 指令错误跳转进入瞬态窗口,
需要用 branch 指令将 BIM 表项训练为 taken;如果希望 branch 指令错误不跳转进入瞬态窗口,则
需要用 branch 指令将 BIM 表项训练为 not taken。\par

\textbf{BRANCH+NONE}的组合,因为 BIM 表项在完成 BIM 的初始化之后都会被初始化为 10,
所以如果需要错误预测跳转,不训练也可以触发。\par

\textbf{STORE/LOAD\_ACCESS\_FAULT/PAGE\_FAULT + NONE}的组合利用了处理器的异常延时检测技术。
这些指令不需要被训练,当处理器执行 load/store 指令时,如果 load/store 指令的地址暂时没有获得,处理器会假设这些指令是没有异常的,
然后乱序执行后续的指令,当 load/store 指令获得地址之后才开始进行权限检查。
因为权限的检测涉及到页表翻译、检查pmp寄存器需要比较长的时间,所以当权限检测错误时有较长的窗口期可以触发瞬态执行漏洞。\par

\textbf{STORE/LOAD\_MISALIGN + NONE}的组合也利用了异常延时检测技术。但是因为地址不对齐可以在得到内存地址之后
立刻进行判断,导致窗口期太短,不容易支持后续的秘密数据泄露操作。\par

\textbf{V4+NONE}的组合利用了内存载入部件 lsu 的预测错误。当进行内存访问时,后执行的 load 指令
假设之前发射的指令没有对需要访问的内存地址进行了修改,直接访问内存获得数据。但是之前发射的 store 指令
在 load 之后修改了该地址的内存,导致 load 之后的指令被回滚,进而打开了瞬态窗口。\par

\textbf{V4+AMO/RETURN/STORE}的组合,这是本工作发现的全新训练组合,
说明 V4 类型需要被额外训练才可以生效,但是具体训练的机制未知,有待后续进一步的研究。\par

根据表\ref{table:trans-summary}可以看到,我们的瞬态漏洞测试程序生成框架在 1 小时内就可以找到 15 种类型的
瞬态窗口,包含了 spectre V1-V4、ret-spectre、meltdown 类型的基本瞬态窗口类型,还发现了‘V4+AMO/STORE/RETURN’的未知情况。
充分证明了我们的生成框架在瞬态窗口的触发上有较高的生成效率和覆盖率。\par

\subsection{完整 PoC 生成评估}

为了验证生成框架生成完整 PoC 的能力和效率,我们让测试框架连续执行完整的 stage1 和 stage2 测试。stage1 执行得到的代码会被
送入 stage2 生成秘密访问和秘密泄露的代码,如果通过差分测试就会被保存起来。因为触发瞬态窗口和秘密泄露的代码是相对独立的,
所以 stage1 得到的代码可以被反复用于 stage2 的代码生成,因此我们让 stage1 得到的每个代码用于 20 次 stage2 代码的生成,
以提高 trigger 代码的利用率,并分摊 trigger reduce 的时间开销。
如此我们让生成框架进行 4 小时的测试,然后分析总结得到的泄露秘密数据的类别和数目。为了可以和 SpecDoctor 的随机代码生成方法进行横向对比,
我们采用和 SpecDoctor 一样对固定触发类型、泄露通道进行定向测试程序生成的测试方法,将我们生成框架的漏洞触发时间和 SpecDoctor 进行横向比较。\par

\subsubsection{泄露秘密数据效率评估}

通过四个小时的测试,生成框架一共进行了 3599 次测试,找到了 166 个完整的 PoC。
因为 PoC 的泄露机制比较繁琐驳杂,需要人工审核,在这里无法对所有 166 个 PoC 的泄露机制进行分门别类,所以
仅对一些有代表性的 PoC 泄露机制进行简要介绍,以佐证我们实验的真实性和完整性。\par

\textbf{低位匹配泄露 secret 数据:}
access\_secret 块在使用高位地址被随机数替换的地址进行内存访问,获得 secret 值时,仍然可以成功获得 secret。实验表明
这首先需要 secret\_migrate 块将数据载入到 cache 或者 line fill buffer,然后因为 buffer 匹配时只使用地址的一部分(高位用于权限验证),
所以仅当低位匹配时,access\_secret 块就可以从 cache 或者 line fill buffer 获得 secret 的值。
实验表明 boom 的 buffer 使用的匹配地址最多不超过 40 位,因此只要访问地址的低 40 位匹配,就可以从 buffer 中泄露 secret。\par

\textbf{通过 cache 侧信道泄露:}
代码样例如图\ref{code:cache-channel}所示,其中 secret 数据被保存在 s0 寄存器中,本体的 secret 值为 101,变体的 secret 值为 0。
当执行该代码时,本体和变体将访问不同地址的数据,进而影响不同位置的 cacheline。
当 decode group 重新执行这部分代码时,因为 s0 的值在本体变体都是 101,所以本体可以在访问时命中 cache,
执行速度比较快,但是变体因为 cache miss,执行速度比较慢,从而产生时间侧信道的差异,进而导致秘密的泄露。\par

\begin{figure}[htbp]
    \centering
    \begin{minted}{c}
        la a1, channel_page_base_0
        andi s0, s0, 0x7f8
        add a1, a1, s0
        flw ft3, 0x5c8(a1)
        ld a7, 0x5e8(a1)
    \end{minted}
    \caption{用cache泄露秘密数据}
    \label{code:cache-channel}
\end{figure}

\textbf{通过阻塞流水线:}
代码样例如图\ref{code:port-stall}所示。因为变体的 s0 等于 0,而本体的等于 101,所以本体会执行这个循环的浮点指令,而变体不会。
当瞬态窗口撤销时,本体发射的浮点指令还在 FPU 中执行无法直接从 FPU 中撤销,所以流水线必须等待 FPU 中的浮点指令执行完毕才可以
回滚瞬态窗口,这导致变体可以比本体提前执行结束。
所以通过阻塞流水线产生了时间差异,进而泄露了 secret 数据。\par

\begin{figure}[htbp]
    \centering
    \begin{minted}{c}
        encode_block_0:
            c.beqz s0, encode_nop_fill
            fsgnj.d ft3, ft8, ft2
            fmsub.d fs10, ft7, ft11, ft9
            fmax.s ft10, fs6, fs6
            fclass.d gp, fs8
            fsub.s ft2, fs1, fs11
            fcvt.d.wu fs1, a5
            fsgnj.s fa3, ft5, ft1
            jal zero, encode_block_0
        encode_nop_fill:
    \end{minted}
    \caption{用阻塞流水线泄露秘密数据}
    \label{code:port-stall}
\end{figure}

另外也包括一些原理不明的 PoC 程序有待后续进一步分析归类,诸如 RSB、BTB、BIM 等前端部件也有可能被作为侧信道。
故而我们的 PoC 至少覆盖了 cache、port 两种侧信道。\par

\subsubsection{SpecDoctor 的横向比较}

这里我们固定了触发瞬态窗口的机制和泄露秘密数据的侧信道,进行定向的 PoC 生成,然后将我们找到第一个 PoC 所使用的时间
和 SpecDoctor 论文使用的时间进行对比,通过横向比较衡量我们生成框架的性能。测试结果如表\ref{table:cmp-specdoctor}所示。\par

\begin{table}[h!]
    \begin{center} 
    \caption{与 SpecDoctor 的性能比较} 
    \label{table:cmp-specdoctor}  
    \resizebox{0.7\linewidth}{!}{
        \begin{tabular}{|l|l|l|l|} 
            \hline
            \textbf{瞬态窗口类型} & \textbf{泄露通道} & \textbf{本框架用时} & \textbf{SpecDoctor 用时}\\
            \hline
            load\_page\_fault   &   dcache      &   2m24s   &   34.7h   \\
            store\_page\_fault  &   dcache      &   1m9s    &   46.1h   \\
            branch\_mispredict  &   dcache      &   2m31s   &   30.6h   \\
            branch\_mispredict  &   front-end   &   2m31s   &   31.2h   \\
            \hline
        \end{tabular}
    }
    \end{center}
\end{table}

可以看到因为 SpecDoctor 使用随机的指令生成方式,所以每个 PoC 生成所需的时间都是几十小时级别的,
而我们使用策略生成+地址敏感测试程序仅仅需要 1-3 min 就可以得到需要的 PoC,这也充分验证了我们生成策略的高效性。\par

\cleardoublepage
\section{总结与展望}

本文提出了一种适用于 RISCV 处理器的瞬态执行漏洞发掘的高效测试程序生成框架。
该框架可以支持物理态、内核态、用户态的执行,支持虚拟地址和物理地址的执行,支持 rv64imafdc 的全集和 zicsr 扩展的子集。
为了解决现有瞬态执行漏洞测试生成框架随机生成的低效性问题,我们采用操作数约束执行、代码块策略生成、
地址敏感的训练代码和泄露代码排布、物理内存切换等技术。该生成测试框架可以在 4 小时内找到 166 个 boom 处理器的完整 PoC,
覆盖 15 种瞬态窗口类型和至少 2 种侧信道,对于给定的窗口类型和侧信道类型可以在分钟的时间量级内找到 PoC,远快于 SpecDoctor 等
随机生成框架的小时级别。\par

虽然与部分瞬态测试程序生成工作相比,本生成框架有更高的生成效率和覆盖率,但是仍有许多不足和改进的空间。\par

\begin{enumerate}
    \item 本生成框架暂不支持使用 csr 指令、rocc 指令、密码学扩展指令、fence 指令等进行训练,在指令的覆盖率上可以进一步扩展。
后续可以考虑为处理器提供 rocc 部件和密码学计算部件,来支持这些指令;对于 csr 指令的修改做一些调整和约束,确保在修改 csr
指令的同时不会破坏程序的正常执行。\par
    \item 本生成框架没有考虑瞬态窗口出现在异常处理程序的情况,故而控制流预测进入异常处理程序的情况会被测试程序忽略掉。
当测试程序发生异常进入异常处理程序时,可能前端的控制流预测组件也会记录这些信息,从而在特定情况下预测进入异常处理程序。
但是生成框架暂时没有对异常处理程序进行插桩和瞬态窗口填充的支持,因此暂时假定瞬态窗口不会在异常处理程序出现。\par
    \item 本生成框架主要考虑的是前端控制流预测错误的情况,对于类似 V4 变体数据流预测错误的情况并没有很好的设计和覆盖。
V4 启示我们并不仅仅只有控制流会发生预测,内存的访问、计算单元的计算结果都可能是预测产生的,因此我们既要考虑控制流预测的情况,
也应该考虑数据流预测的情况。\par
    \item 本生成框架暂时不支持中断发生的情况。中断会对处理器的执行情况进行改变,进而影响处理器控制流的预测、微架构的编码
等,对于瞬态窗口触发和秘密数据泄露也许都会有影响。因此需要提供通过监视处理器内部执行情况触发外部中断的机制,让中断在合适的时机发生。\par
    \item 本生成框架在瞬态漏洞定位方面存在不足。虽然生成框架可以产生 PoC,但是 encode 块内部的代码过于繁多和驳杂,依靠手动定位和
分析过于费时费力,所以需要一个简单有效的机制过滤掉 encode 块中的冗余代码,将搜索空间变得尽可能小,进而提高漏洞定位的效率。\par
    \item 本生成框架仅在 boom 处理器上进行测试验证,对于 xiangshan 等其他乱序处理器、大型商用处理器都没有进行验证考量,
其生成功能的通用性、泛化性还有待更多实验的验证。\par
\end{enumerate}

在后续的工作中,我们会对这些方面进行进一步的扩展,力求实现功能更加完备的 RISCV 处理器瞬态漏洞测试程序生成框架。\par


