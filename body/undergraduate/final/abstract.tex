\cleardoublepage{}
\begin{center}
    \bfseries \zihao{3} 摘~要
\end{center}

随着侧信道攻击技术的发展,瞬态攻击漏洞已经成为现代处理器安全的关键问题。
虽然近年来一些基于 RISCV 体系结构的瞬态执行漏洞挖掘工作已经取得了出色的成果,
但是因为瞬态执行漏洞的复杂性,瞬态执行漏洞测试程序自动化生成仍然十分困难。
目前的生成框架普遍存在程序数据流、控制流约束弱,瞬态窗口触发效率低下,代码块随机堆砌、难以协同工作等问题。
因此如何自动化地高效生成兼具有效性和多样性的瞬态执行漏洞测试程序,仍然是该领域亟待解决的重要课题。\par

本文提出的针对 RISCV 处理器的瞬态执行漏洞高效测试程序生成框架对上述三个问题进行了解决。
首先我们针对瞬态执行的场景设计了控制流、数据流的约束条件,并引入具体值求解技术。
为解决瞬态窗口触发效率问题,我们为预设的瞬态窗口触发指令构造训练代码,
并引入硬件可切换内存单元解决训练代码地址对齐问题。为解决代码块协同工作问题,
我们将瞬态执行的主体代码进行分块策略生成,并有序组装成完整程序。
该生成框架,最终可以支持M、S、U态的测试程序执行,支持虚拟地址和物理地址的执行,
支持 rv64imafdc 的全集和 zicsr 扩展的子集,可以在 4 小时内覆盖 BOOM 处理器
 15 种瞬态窗口类型和至少 5 种侧信道。\par

\textbf{关键词:}RISC-V;处理器测试;瞬态执行漏洞;测试程序生成

\cleardoublepage{}
\begin{center}
    \bfseries \zihao{3} Abstract
\end{center}

With the development of side-channel attack techniques, 
transient execution vulnerabilities have become a critical vulnerability
in modern processor security. In recent years, 
some research on RISC-V architecture-based transient execution vulnerability detection 
has achieved excellent results,
but, due to the complexity of transient execution vulnerabilities, 
it is still extremely difficult to automatically generate testing programs for transient execution vulnerabilities.
Current generation frameworks often suffer from weaknesses 
such as data flow and control flow constraints, 
low trigger efficiency, difficulties in collaborative work among code blocks. 
Therefore, how to generate effective and diverse transient execution vulnerability testing programs 
efficiently and automatically remains a crucial unsolved problem in this field.\par

This work proposes an efficient testing program generation framework for RISCV processors' transient execution vulnerabilities
to solve three existing problems mentioned.
Firstly, we designed scenario-based constraint conditions for control flow and data flow of transient executions, 
and introduced concrete value-solving techniques.
To solve the issue of low trigger efficiency for transient windows, 
we constructed training code for preset transient window trigger instructions 
and introduced hardware-switchable memory units 
to resolve the training code's address alignment issues.
To solve the problem of collaborative work among code blocks, 
we employed a block-based strategy to generate the main body of transient execution code 
and assembled it into a complete program in an orderly manner.
This generation framework ultimately supports testing programs executed in M, S, U modes, 
as well as virtual and physical address. 
It also supports the full set of rv64imafdc instructions and a subset of zicsr extensions. 
It can cover 15 types of transient windows and at least 5 types of side channels of BOOM processors within 4 hours.\par

\textbf{Keywords:}RISC-V;processor test;transient execution vulnerability;testing program generation


