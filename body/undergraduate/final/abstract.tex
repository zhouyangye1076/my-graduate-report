\cleardoublepage{}
\begin{center}
    \bfseries \zihao{3} 摘~要
\end{center}

随着大量复杂投机执行技术的引入和侧信道攻击技术的发展,瞬态攻执漏洞已经成为现代处理器安全的关键问题。
近年来一些基于 RISCV 体系结构的瞬态执行漏洞挖掘工作已经取得了出色的成果,并成功找到了处理器中真实存在的瞬态执行漏洞。
但是因为瞬态执行漏洞本身语义和执行流程的复杂性,使得针对瞬态执行漏洞测试程序自动化生成变得十分困难。
目前的生成框架多存在程序数据流、控制流约束弱,瞬态窗口触发效率低下,代码块随机堆砌、难以协同工作等问题。
因此如何自动化地高效生成兼具有效性和多样性的瞬态执行漏洞测试程序,仍然是瞬态漏洞模糊测试领域亟待解决的重要课题。\par

本文提出的针对 RISCV 处理器的瞬态执行漏洞高效测试程序生成框架对上述三个问题进行了解决。
首先我们针对瞬态执行的场景设计了控制流、数据流的约束条件,并通过修改 spike 模拟器引入具体值求解技术。
为解决瞬态窗口触发效率问题,我们对于预设的瞬态窗口地址和触发瞬态窗口的指令有针对性地生成训练代码,
并引入硬件可切换内存单元,通过训练代码和被训练代码共用物理地址的方式提高训练效果。为解决代码块协同工作问题,
我们将瞬态执行的主体代码细分为十个功能块,根据语义对每个功能块进行策略生成,并有序组装成完整程序。
该生成框架,最终可以支持物理态、内核态、用户态的测试程序执行,支持虚拟地址和物理地址的执行,
支持 rv64imafdc 的全集和 zicsr 扩展的子集。该生成测试框架可以在 4 小时内找到 166 个 BOOM 处理器的完整 PoC,
覆盖 15 种瞬态窗口类型和至少 5 种侧信道,对于给定的窗口类型和侧信道类型可以在 min 的时间量级内找到 PoC,远快于 SpecDoctor 等
随机生成框架。\par

\textbf{关键词:}RISC-V;处理器测试;瞬态执行漏洞;测试程序生成


\cleardoublepage{}
\begin{center}
    \bfseries \zihao{3} Abstract
\end{center}

With the development of complex speculative execution technologies 
and side-channel attack techniques, transient execution vulnerabilities have become a critical vulnerability
in modern processor security. In recent years, 
some research on RISC-V architecture-based transient execution vulnerability detection 
has achieved excellent results and identified real-world transient execution vulnerabilities in processors successfully.
However, due to the complexity of transient execution vulnerabilities, 
it is extremely difficult to automatically generate testing programs for transient execution vulnerabilities.
Current generation frameworks often suffer from weaknesses 
such as data flow and control flow constraints, 
low trigger efficiency, difficulties in collaborative work among code blocks. 
Therefore, how to generate effective and diverse transient execution vulnerability testing programs  efficiently and automatically 
remains a crucial unsolved problem in the field of transient vulnerability fuzz testing.\par

This work proposes an efficient testing program generation framework for RISCV processors' transient execution vulnerabilities
to solve three existing problems mentioned.
Firstly, we designed scenario-based constraint conditions for control flow and data flow of transient executions, 
and introduced concrete value-solving techniques by modifying the Spike simulator.
To solve the issue of low trigger efficiency for transient windows, 
we generated training code for predetermined transient window addresses and instructions that trigger transient windows.
We also introduced a hardware-switchable memory unit,
which improves training effectiveness by sharing physical addresses between training code and trained code.
To solve the problem of collaborative work among code blocks, 
we divided the main body of transient execution code into ten functional blocks, 
generated each block strategically based on semantics, 
and assembled them to form a complete program.
This generation framework ultimately supports testing programs executed in physical, kernel, and user modes, 
as well as virtual and physical address executions. 
It also supports the full set of rv64imafdc instructions and a subset of zicsr extensions. 
This testing framework can find 166 complete PoCs for BOOM processors within 4 hours, 
covering 15 types of transient windows and at least 5 types of side channels. 
For given window types and side channel types, it can find PoCs in minimal time scales, 
significantly outperforming random generation frameworks like SpecDoctor.\par

\textbf{Keywords:}RISC-V;processor test;transient execution vulnerability;testing program generation


